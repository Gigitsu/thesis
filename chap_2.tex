In questo capitolo introduciamo notazioni formali e diamo una breve spiegazione del funzionamento delle reti neurali e degli strumenti utilizzati.

\section{Il Tempo}

Le RNN non sono limitate a sequenze indicizzate in maniera temporale.
Sono state usate con successo anche per seqenze di dati non temporali, come ad esempio i dati genetici.
In ongi caso, la computazione procede nel tempo e molte importanti applicazioni hanno un aspetto temporale esplicito o implicito.

Nonostante, in questa tesi, ci riferiremo al tempo i metodi descritti sono applicabili ad una famiglia pi\`u ampia di compiti.
Parlando di tempo ci riferiamo ad un campione $x^{(t)}$ in input e ad un valore atteso $y^{(t)}$ in output che sono generati in \emph{sequenze di fasi temporali} discrete indicate da $t$
La nostra sequenza pu\`o essere formata da un numero finito di campioni o da un numero infinito ma numerabile di campioni.
Se abbiamo a che fare con un numero finito di campioni allora possiamo indicare con $T$ il massimo indice temporale.
Quindi una sequenza di valori di input consecutivi pu\`o essere scritta come $(x^{(1)}, x^{(2)}, \dots, x^{(T)})$ mentre gli output come $(y^{(1)}, y^{(2)}, \dots, y^{(T)})$.
Questi valori possono essere dei campioni, presi ad intervalli regolari di tempo, di un processo reale continuo come ad esempio i fotogrammi che compongono un video.
Gli intervalli di tempo possono anche essere dei semplici valori ordinali senza una durata esatta.
\`E il caso, ad esempio, delle sequenze genetiche che hanno un ordine ma non un ordine temporale o ancora del linguaggio naturale dove le parole hanno un ordine logico ben preciso che, tuttavia, non corrisponde ad intervalli di temporali ben defniti.
Ad esempio nella frase ``\emph{Lisa suona il sassofono}'' abbiamo che $x^{(1)}$ = Lisa, $x^{(2)}$ = suona, ecc. Ciascuna parola corrisponde ad intervalli di tempo che non sono costanti, ``\emph{il}'' e ``\emph{sassofono}'' hanno bisogno di tempi diversi per essere pronunciati.

\section{Reti Neurali}
Le reti neurali sono modelli computazionali ispirati dalla biologia del sistema nervoso centrale.
Generalmente una rete neurale \`e formata da un insieme di \emph{neuroni artificiali}, comunemente chiamati \emph{nodi} o \emph{unit\`a}, collegati da un insieme di archi diretti che, intuitivamente, rappresentano le \emph{sinapsi} di una rete neurale biologica.
Associato ad ogni neurone $j$ vi \`e una funzione di attivazione $l_j$, chiamata anche funzione di collegamento (o funzione link).
In questa tesi user\`o la notazione ``$l_j$'' invece di ``$h_j$'' (notazione usata in altri documenti) per distinguere la funzione di attivazione $l_j$ dal valore dei nodi nascosti in una rete, che vengono comunemente indicati con \textbf{h} nella letteratura delle RNN.

Associato ad ogni arco dal nodo $j^{'}$ al nodo $j$ vi \`e un peso $w_{jj^{'}}$. Seguento la convenzione adottata in molti altri documenti che trattano reti neurali, indicheremo i neuroni con $j$ e $j^{'}$ mentre, con $w_{jj^{'}}$, indicheremo il peso corrispondente all'arco diretto che parte dal nodo $j^{'}$ e arriva al nodo $j$.
\`E importante notare che in altri documenti e libri, come ad esempio su Wikipedia, gli indici dei pesi sono invertiti e che $w_{j^{'}j} \neq w_{jj^{'}}$ indica il peso sull'arco diretto dal nodo $j^{'}$ al nodo $j$.

Il valore $v_j$ di ciascun neurone $j$ \`e calcolato applicando la sua funzione di attivazione ad una somma pesata dei suoi valori di input (\mytodo{Aggiungere un'immagine!}): %TODO pensare ad un titolo piu' appropiato
\begin{equation} % \begin{equantion*} per non numerare l'equazione
  v_j = l_j\left( \sum_{j^{'}} w_{jj^{'}} \cdot v_{j^{'}} \right)
\end{equation}
Per comodit\`a, ci riferiremo alla somma pesata all'interno delle parentesi come l'\emph{attivazione in arrivo} e la indicheremo con $a_j$. Rappresentiamo l'intero processo in figura disegnando i neuromi coe dei cerchi mentre gli archi come delle frecce che li collegano.
Quando possibile, verr\`a utilizato un simbolo per indicare l'esatta funzione di attivazione utilizzata, ad esempio $\sigma$ per la fnzione sigmoide.

Scelte abbastanza comuni per la funzione di attivazione includono la funzione sigmoide $\sigma(z) = 1/(1+e^{-z})$ e la funzione \emph(tanh) $\phi(z)=(e^z-e^{-z})/(e^z+e^{-z})$ che \`e diventata molto comune nelle reti neurali di tipo \emph{feedforward} ma \`e stata utilizzata anche in reti neurali ricorrenti~\cite{Sutskever:2011}.
Un'altra funzione di attivazione che \`e diventata lo stato dell'arte nella ricerca di deep learning \`e la funzione ReLU (\emph{rectified linear unit}) $l_j(z)=\operatorname{max}(0, z)$.
Questa funzione ha dimostrato di poter migliorare le prestazioni di molte reti neurali in una grande variet\`a di applicazioni, che spaziano dal riconoscimento vocale al riconoscimento di oggetti, ed \`e stata utilizzata anche in reti neurali ricorrenti~\cite{Bengio:2013}.

La funzione di attivazione da applicare sui nodi di output dipende dall'applicazione.
Per una classificazione a pi\`u classi, applichiamo al livello di output una funzione non lineare softmax.
La funzione softmax calcola l'output come:
\begin{equation}
  \hat{y_k} = \frac{e^{a_k}}{\sum_{k^{'}=1}^{K} e^{a_{k^{'}}}}
\end{equation}
dove $K$ \`e il numero totale di possibili output (classi). Il denominatore \`e una funzione di normalizzazione che consiste nella somma di funzioni esponenziali dei valori dati in output da tutti i nodi e serve per assicrarsi che l'output totale sommi ad 1.
Nel caso di regressioni invece viene comunemente utilizzata una funzione lineare come output.
Dato che nella maggioranza dei casi le reti neurali, specialmente quelle ricorrenti, vengono utilizzate per applicazioni che coinvolgono la classificazione, durante questa tesi, a meno che diversamente specificato, daremo per scontato l'uso della funzioen softmax come output.

\section{Reti Neurali Feedforward}
Con un modello computazionale a rete neurali, \`e necessario determinare l'ordine con cui la computazione dovrebbe procedere.
I nodi dovrebbero essere calcolati uno alla volta e poi aggiornati, oppure i valori di tutti i nodi dovrebbero essere calcolati iniseme per poi applicare tutti gli aggiornamenti simultaneamente?
Le \emph{reti neurali feedforward} (\mytodo{aggiungere figura ffnn}) sono una classe ristretta di reti neurali che affrontano questo prolema proibnedo i cicli dal grafo delle connessioni neurali. %TODO aggiungere immagine
In questo modo tutti i nodi possono essere disposti in livelli.
I valori in output di ciascun livello possono essere calcolati solo a partire dai valori di output dei livelli precedenti.

L'input $x$ viene dato in pasto ad una rete neurale feedforward impostando i valori dei nodi del livello pi\`u in basso.
I valori dei nodi di ciascuno dei livelli superiori non potranno essere calcolati finch\`e non saranno disponibili i valori in output $\hat{y}$ dei livelli inferiori.
Queste tipologie di reti sono usate di frequente per applicazioni di apprendimento supervisionato come classificazione e regressione.
L'apprendimento \`e ottenuto aggiornando iterativamente i pesi dei singoli archi in modo da minimizzare una funzione di perdita, $\mathcal{L}(\hat{y},y)$, che penalizza la distanza fra l'output desiderato $y$ e l'output predetto $\hat{y}$ tramite tecniche di ottimizzazione.
Nonostante l'algoritmo di ottimizzazione esatto \`e un noto problema NP-Completo, una grande quantit\`a di euristiche pre addestramento e avanzate tecniche di ottimizzazione hanno condotto ad un impressionante numero di successi empirici su molte applicazioni di apprendimento supervisionato.

L'algoritmo utilizzato con maggiore successo per addestrare una rete neurale \`e l'algoritmo di backpropagation, introdotto da Rumelhart et al. nel 1985~\cite{Rumelhart:1985}.
Questo algoritmo usa la regola della catena per calcolare la derivata di una funzione di perdita $\mathcal{L}$ rispetto ciascun parametro nella rete.
I pesi sui singoli archi vengono poi tramite la discesa dei gradienti.
Dato che la superficie di perdita non \`e convessa non vi \`e alcuna garanzia che l'algoritmo di backpropagation riesca a trovare un minimo globale.
Ci\`o nonostante, nella pratica, reti addestrate in questo modo hanno ottenuto notevoli successi.

Tuttavia le reti feedforward sono limitate.
Dopo che ciascun campione \`e stato processato l'intero stato della rete viene perso.
Se i campioni sono indipendenti gli uni dagli altri questo non presenta assolutamente un problema.
Ma se i dati sono in una relazione temporale, questo non \`e accettabile.
I fotogrammi di un video o le parole di una frase rappresentano situazioni in cui l'assunzione di indipendenza fallisce.

\section{Reti Neurali Ricorrenti}
Le \emph{reti neurali ricorrenti} costituiscono un sovrainsieme proprio delle reti neurali feedforward che, a differenza di quest'ultime, includono degli archi ricorrenti.
Questi archi ricorrenti si estendono su intervalli temporali adiacenti e introducono il concetto di tempo nel modello.
Mentre le RNN possono non contenere cicli tra gli archi convenzionali, gli archi ricorrenti possono formare cicli.
Al tempo $t$, i nodi che ricevono un input da un arco ricorrente, ricevono un \emph{input di attivazione} sia dal campione corrente $x^({t})$ che dai nodi nascosti $h^{(t-1)}$ del precedente stato della rete.
L'output $\hat(y^{(t)})$ \`e poi calcolato in base allo stato nascosto $h^{(t)}$ del tempo $t$.
Quindi, l'input $x^({t})$ al tempo $t-1$ pu\`o influenzare l'output $\hat(y^{(t)})$ al tempo $t$ proprio grazie a queste connessioni ricorrenti.

Le seguenti due equazioni, mostrano i calcoli necessari per eseguire, per ogni fase temporale, il passaggio in avanti dei dati di una semplice rete neurale ricorrente:
\begin{equation}
  h^{(t)} = \sigma(W_{hx}x + W_{hh}h^{(t-1)} + b_h)
\end{equation}
\begin{equation}
  \hat{y}^{(t)} = \operatorname{softmax}(W_{yh}h^{(t)} + b_y)
\end{equation}

dove $W_{hx}$ \`e la matrice dei pesi tra il livello di input e il livelo nascosto mentre $W_{hh}$ \`e la matrice dei pesi ricorrenti fra i livelli nascosti di due fasi temporali adiacenti.
I vettori $b_h$ e $b_y$ rappresentano uno scostamento (\emph{bias}) che permettono a ciascun nodo di apprendere un offset.

I modelli discussi in questa tesi consistono in reti con livelli nascosti ricorrenti.
Tuttavia, sono stati proposti modelli, come la rete di Jordan, che ammettono la presenza di connessioni tra gli output della rete in uno stato e il livello nascosto della rete nello stato successivo.

Una semplice rete neurale \`e mostrata in \mytodo{aggiungere immagine rnn}. %TODO aggiungere immagine
La dinamica di questa rete attravesso pi\`u fasi temporali pu\`o essere visualizzata \emph{dispiegando} la rete (\mytodo{aggiungere immagine}). %TODO aggiungere immagine
Con questa visualizzazione, il modello pu\`o essere interpretata come una rete non ciclica, ma piuttosto come una rete con un livello per intervallo di tempo a dei pesi condivisi tra gli intervalli temporali.
Diventa quindi chiaro come una rete dispiegata in questo modo pu\`o essere addestrata attraverso pi\`u fasi temporali usando l'algoritmo di backpropagation.
Questo algoritmo viene chiamato \emph{backpropagation through time} (BPTT, backpropagation attraverso il tempo), ed \`e stata introdotta nel 1990~\cite{Werbos:1990}

\subsection{Addestramento}
L'apprendimento con le reti neurali ricorrenti \`e stato a lungo visto come qualcosa di difficile.
Cos\`i come per tutte le reti neurali, l'ottimizzazione della funzione di perdita \`e un problema NP-Completo.
Ma l'apprendimento con le reti ricorrenti pu\`o essere reso ancora pi\`u complesso a causa della difficolt\`a nell'apprendere delle dipendenze a lungo raggio.
Il noto problema della \emph{scomparsa} ed \emph{esplosione} dei gradienti si verifica quando l'errore viene propagato attraverso molte fasi temporali.
\mytodo{dire di pi\`u a riguardo?} L'impatto dell'input al tempo $\mathcal{T}$ sull'output al tempo $t$ esploder\`a esponenzialmente oppure raggiunger\`a rapidamente zero al crescere di $\mathcal{T} - t$, a seconda se il peso $\abs{w_{jj}}>1$ oppure $\abs{w_{jj}}<1$ ma anche in base alla funzione di attivazione utilizzata %TODO dire di piu
(ad esempio con una funzione di attivazione $l_j = \sigma$ si verificher\`a maggiormente il problema della sparizione del gradeinte, viceversa con la funzione ReLU $\operatorname{max}(0, x)$ il gradiente esploder\`a).

Una possibile soluzione al problema consiste nell'usare una versione leggermente modificate dell'algoritmo BPTT che prende il nome di \emph{truncated backpropagation through time} (TBPTT)~\cite{Williams:1989}.
Con l'algoritmo TBPTT viene impostato un valore che indicia il numero massimo di temporali lungo le quali pu\`o essere propagato l'errore.
In questo modo si attenua il problema dell'esplosione del gradiente perdendo, tuttavia, la capacit\`a di apprendere dipendenze a lungo raggio.

Il problema dell'ottimizzazione rappresenta un fondamentale ostacolo che non pu\`o essere risolto semplicemente modificando l'architettura della rete.
\`E noto dal 1993 che ottimizzare una rete neurale di anche solo 3 livelli costituisce un problema NP-Completo.
Tuttavia, recenti studi sia teorici che empirici, suggeriscono che il problema non \`e cos\`i insormontabile nella pratica come si potrebbe pensare.

Inoltre implementazioni sempre pi\`u performanti ed migliorate euristiche per il calcolo dei gradienti hanno reso l'addestramento delle RNN fattibile.
Ad esempio, implementazioni degli algorotmi di forward a backward propagation che sfruttano la GPU, come Theano e Torch (strumento utilizzato per gli esperimenti di questa tesi), hanno semplificato la realizzazione di veloci algoritmi di apprendimento.

\subsection{Architetture moderne}
Sono due le architetture RNN di maggior successo per l'apprendimento di dati sequenziali e risalgono entrambe al 1997.
La prima, \emph{Long Short-Term Memory}, ideata da Hochreiter e Schmidhuber, introduce il concetto di cella di memoria, un'unit\`a computazionale che rimpiazza il tradizionale neurone artificiale nei livelli nascosti della rete.
Con queste celle di memoria, la rete \`e in grado di superare le difficolt\`a incontrate dalle precedenti implementazioni durante la fase di apprendimento.
La seconda, \emph{Bidirectional Recurrent Neural Network}, ideata da Schuster e Paliwal, introduce invece l'architettura BRNN nella quale, per calcolare l'output del tempo $t$, vengono usate tanto le informazioni provenenti dal passato quanto quelle provenienti dal futuro.
Questo approccio \`e in contrasto con i sistemi precedenti, dove solo gli input provenienti da fasi temporali passate potevano influenzare gli output.
Fortunatamente, le due architetture non sono mutuamente esclusive e sono state combinate con successo.

Nel corso di questa tesi ci concentreremo sull'architettura \emph{Long Short-Term Memory}.

\section{Long Short-Term Memory (LSTM)}
Nel 1997, per superare il problema della scomparsa del gradiente, Hochreiter e Schmidhuber introdussero il modello LSTM.
Questo modello somiglia ad una rete neurale standard con un livello nascosto ricorrente, con l'unica differenza che i normali nodi di un livello nascosto \mytodo{riferimento all'immagine della rete neurale semplice} sono rimpiazzati da celle di memoria \mytodo{immagine}. %TODO immagini
La cella di memoria contiene un nodo sul quale insiste un un arco ricorrente connesso a se stesso con peso pari ad 1, assicurandosi, in questo modo, che il gradiente possa passare attraverso molte fasi temporali senza scomparire o esplodere.

Per distinguere la presenza di una cella di memoria da un nodo normale, indicheremo queste celle con $c$.

Il termine ``Long Short-Term Memory'' deriva dalla seguente intuizione.
Le reti neurali ricorrenti pi\`u semplici hanno una \emph{memoria a lungo termine} (\emph{long term memory}) sotto forma di pesi.
I pesi, durante l'apprendimento, cambiano molto lentamente, codificando, in questo modo, la conoscenza.
Queste hanno anche una \emph{memoria a breve termine} (\emph{short term memory}) sotto forma di attivazioni effimere, che passano dall'output di ciascun nodo nei nodi successivi.
Il modello LSTM introduce una sorta di memoria intermedia tramite l'uso delle celle di memoria.
Una cella di memoria \`e una composizione di unit\`a pi\`u semplici con la nuova aggiunta di nodi moltiplicativi, indicati nel diagramma con il simbolo $\prod$.
Di seguito sono descritti tutti gli elementi che compongono una cella di memoria.
\begin{itemize}
  \item \emph{Internal State:} Il cuore di ogni cella di memoria \`e un nodo $s$ con una funzione di attivazione lineare.
  Dato che indichiamo con $c$ una cella di memoria, il suo stato interno sar\`a indicato con $c_s$.
  \item \emph{Constant Error Carousel:} Lo stato interno $c_s$ ha un arco connesso a se stesso (ricorrente) con peso pari ad 1.
  Quest'arco, chiamato \emph{constant error carousel}, si estende tra fasi temporali adiacenti con un peso costante, questo assicura che l'errore possa passare attraverso le fasi temporali senza sparire o esplodere.
  \item \emph{Input Node:} Questo nodo si comporta come un normale nodo, prendendo gli input tanto dal resto della rete alla fase temporale precedente quanto dagli input al tempo corrente.
  Solitamente questo nodo viene indicato con $g$ e questa tesi non far\`a eccezione, anche se potrebbe generare confusione dato che sarebbe pi\`u appropiato usare $g$ per indicare i cancelli.
  \`E importante notare che quando usiamo la notazione vettoriale ci riferiamo ad un intero livello di celle.
  Ad esempio, $g$ \`e un vettore che contiene i valori di $g_c$ di tutte le celle in un dato livello.
  Quando, invece, usiamo il pedice $c$, allora ci riferiamo ad una singola cella di memoria.
  \item \emph{Multiplicative Gating:} I ``cancelli'' moltiplicativi (gates) sono una peculiarit\`a dei modelli LSTM.
  Qu\`i un'unit\`a sigmoidale chiamata \emph{gate} viene addestrata dato l'input e una connessione ricorrente in arrivo dal passo temporale precedente.
  Alcuni valori di interesse sono poi moltiplicati per l'output di questa unit\`a.
  Se il cancello da in output 0, allora questo \`e chiuso e il flusso dei dati viene interrotto.
  Se, invece, da in output 1 allora in cancello \`e aperto e il flusso dei dati pu\`o passarci attraverso.
  Il modello LSTM originale prevedeva due cancelli:
  \begin{itemize}
    \item \emph{Input Gate:} Il primo \`e l'\emph{input gate} $i_c$, che \`e moltiplicato per il nodo input $g_c$
    \item \emph{Output Gate:} Il secondo cancello \`e chiamato \emph{output gate} $o_c$.
    Questo viene moltiplicato per il valore dello stato intenro $s_c$ per produrre il valore in output $v_c$ della cella di memoria.
    Questo poi viene dato in pasto al livello nascosto della LSTM nel prossimo passo temporale $h^{(t+1)}$ insieme all'output $\hat{y}^(t)$ generato al passo corrente.
  \end{itemize}
  Questi cancelli vengono indicati con $y_{in}$ e $y_{out}$, tuttavia questa notazione genera confusione perch\`e $y$ \`e solitamente utilizzato per indicare l'output nella letteratura di machine learning.
  Per questo motivo utilizzeremo le notazioni $i$, $o$ ed $f$ per indicare i cancelli di \emph{input}, \emph{output} e \emph{forget} (di cui pareler\`o a breve).
\end{itemize}

Fin da quando \`e stato proposto il modello LSTM originale, molte varianto sono state proposte.
Il \emph{forget gate}, proposti nel 2000 da Gers e Schmidhuber~\cite{Gers:2000}, aggiunge un cancello simile a quelli di input e output che permette, alla rete neurale, di svuotare le informazioni presenti nel \emph{constant error carousel}.
In altre parole, questo ulteriore cancello da alla LSTM la capacit\`a di apprendere quando ``dimenticare'' le informazioni ottenute dalle precendit fasi temporali.
Il \emph{forget gate} \`e diventato di uso comune, un pilastro dei sucessivi lavori sulle LSTM.

Formalmente, la computazione in una LSTM procede in accordo ai seguenti calcoli che devono essere valutati ad ogni fase temporale.
Queste equazioni forniscono l'algoritmo completo per una moderna LSTM, comprensiva di forget gate.
\begin{equation}
  g^{(t)} = \phi(W_{gx}x^{(t)} + W_{ih}h^{(t-1)} + b_g)
\end{equation}
\begin{equation}
  i^{(t)} = \sigma(W_{ix}x^{(t)} + W_{ih}h^{(t-1)} + b_i)
\end{equation}
\begin{equation}
  f^{(t)} = \sigma(W_{fx}x^{(t)} + W_{fh}h^{(t-1)} + b_f)
\end{equation}
\begin{equation}
  o^{(t)} = \sigma(W_{ox}x^{(t)} + W_{oh}h^{(t-1)} + b_o)
\end{equation}
\begin{equation}
  s^{(t)} = g^{(t)} \odot i^{(t)} + s^{(t-1)} \odot f^{(t)}
\end{equation}
\begin{equation}
  h^{(t)} = s^{(t)} \odot o^{(t)}
\end{equation}

dove $\odot$ sta per moltiplicazione elemento per elemento.
Il calcolo di una LSTM semplice, senza forget gate, \`e dato impostando $f^{(t)} = 1$ per ogni $t$.
Seguendi l'ultimo stato dell'arte abbiamo usato la funzione tanh $\phi$ per i nodi di input $g$, nonostante l'implementazione originale del modello LSTM prevedesse l'uso di una funzione sigmoide $\sigma$.
Ancora, $h^{(t-1)}$ \`e un vettore contenente i valori $v_c$ dati in output da ciascuna cella di memoria $c$ del livello nascosto dalla precedente fase temporale.

Intuitivamente, in termini di forward pass dei dati, il modello LSTM pu\`o apprendere quando lasciar passare l'attivazione nello stato interno.
Finttantoch\`e il cancello di input (\emph{input gate}) resta chiuso (da in output 0), nessuna attivazione pu\`o passare.
Allo stesso modo, il cancello di output (\emph{output gate}) apprende quando lasciar uscire il valore dello stato intenro.
Quando entrambi i cancelli sono \emph{chiusi}, l'attivazione rimane confinata, senza subire modifiche ne, tantomeno, influenzare l'output delle fasi temporali intermedie.
In terminidi backward pass, il \emph{constant error carousel} permette al gradiente di poter essere propagato attraverso molte fasi temporali, senza sparire o esplodere.
In quest'ottica, quindi, i cancelli apprendono quando lasciar entrare l'\emph{errore} e quando lasciarlo uscire.
Nella pratica, il modello LSTM ha dimostrato un'abilit\`a superiore nell'apprendere dinepndenze a lungo raggio, rispetto alle normali RNN.
Di conseguenza, questo modello, \`e diventato l'attuale stato dell'arte nel campo delle reti neurali.
