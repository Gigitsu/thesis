In questo capitolo parler\`o dei dataset utilizzati e di come \`e stato impostato l'esperimento.

\section{Part-of-Speech tagging}
In linguistica, il PoS tagging, \`e un processo che consisite nel raggruppare le parole di una frase in calssi dette, appunto, \emph{Part of Speech} o \emph{classi morfologiche}.
Questo raggruppamento, viene fatto sia in base alla definizione della parola stessa che al contesto in cui si trova.

Nella grammatica tradizionale, esistono un numero limitato di classi morfologiche (sostantivo, verbo, aggettivo, articolo, pronome, avverbo, congiunzione, preposizione, ecc..).

Ad esempio, possiamo classificare la frase \emph{Il cane abbaia.} in questo modo:

\centerline{Il\textbf{/ART} cane\textbf{/SOST} abbaia\textbf{/V} .\textbf{/PUNT}}

Tuttavia, chiaramente, esistono molte pi\`u classi di queste.
Per il pronome, possiamo trattare le forme singolari, plurali e possessive come classi distinte.
In molti linguaggi, inoltre, le parole possono essere distinte in base al loro ``caso'' o in base al genere.

Un esempio di classificazione pi\`u dettagliata pu\`o essere:

\begin{center}
Il\textbf{/ART:m:s}

cane\textbf{/SOST:m:s}

abbaia\textbf{/V:ind:pr:3:s}

.\textbf{/PUNT:sent}
\end{center}

In linguistica sono previste classi morfologiche per vari livelli di dettaglio, in base al modello di classificazione scelto.

Solitamente, in sistemi di PoS tagging computerizzati, vengono adottati modelli di classificazione che prevedono un numero elevato di classi (da 50 o pi\`u) e variano in base alla lingua adottata.
Esistono vari modelli comunemente accettati, quelli utilizzati per gli esperimenti di questa tesi sono:
\begin{itemize}
  \item Penn Treebank Tagset, composto da 42 classi e usato per la lingua inglese.
  \item Tanll Tagset, che conta fino a 328 classi ed \`e usato per la lingua italiana.
\end{itemize}

\section{Dataset utilizzati}
\nocite{Zanchetta:2005}
\nocite{Attardi:2008}

\subsection{CoNLL 2000}
CoNLL (\emph{Conference on Natural Language Learning}), \`e una conferenza organizzata annualmente dal gruppo \emph{SIGNLL}, a partire dal 1999.
Per ogni edizione della conferenza, sono state proposte delle attivit\`a condivise, ciascuna delle quali comprendeva dati di test e di training forniti direttamente dagli organizzatori.
In questo modo i partecipanti potevano essere valutati e confrontati in maniera sistematica.

Per gli esperimenti di questa tesi, sono stati usati i dataset do training e test forniti durante la quarta edizione della conferenza, tenutasi nel 2000.
I dati forniti consistono in file di testo dove ogni riga corrisponde ad una parola, mentre una riga vuota denota la fine di una frase.
Ciascuna riga \`e costituita, a sua volta, da 3 colonne:
\begin{itemize}
  \item la prima colonna contiene la parola corrente
  \item la seconda colonna la classe morfologica della parola
  \item la terza riga contiene un tag che indica la parte della frase a cui, la parola corrente, appartiene
\end{itemize}

Di seguito un esempio:

\begin{center}
  \begin{minipage}{5cm}
    \begin{verbatim}
     He        PRP  B-NP
     reckons   VBZ  B-VP
     the       DT   B-NP
     current   JJ   I-NP
     account   NN   I-NP
     deficit   NN   I-NP
     will      MD   B-VP
     narrow    VB   I-VP
     to        TO   B-PP
     only      RB   B-NP
     #         #    I-NP
     1.8       CD   I-NP
     billion   CD   I-NP
     in        IN   B-PP
     September NNP  B-NP
     .         .    O
    \end{verbatim}
  \end{minipage}
\end{center}

Di seguito la lista completa di classi del modello PennTreebank (Tabell~\ref{tab:penn-tagset}):

\begin{longtable}{| c | p{.40\textwidth} | p{.50\textwidth} |} \hline
  \thead{Tag} & \thead{Descrizione} & \thead{Esempio} \\ \hline
  CC & conjunction, coordinating & and, or, but \\ \hline
  CD & cardinal number & five, three, 13\% \\ \hline
  DT & determiner & the, a, these  \\ \hline
  EX & existential there & there were six boys  \\ \hline
  FW & foreign word & mais  \\ \hline
  IN & conjunction, subordinating or preposition & of, on, before, unless  \\ \hline
  JJ & adjective & nice, easy \\ \hline
  JJR & adjective, comparative & nicer, easier \\ \hline
  JJS & adjective, superlative & nicest, easiest  \\ \hline
  LS & list item marker &   \\ \hline
  MD & verb, modal auxillary & may, should  \\ \hline
  NN & noun, singular or mass & tiger, chair, laughter  \\ \hline
  NNS & noun, plural & tigers, chairs, insects  \\ \hline
  NNP & noun, proper singular & Germany, God, Alice  \\ \hline
  NNPS & noun, proper plural & we met two Christmases ago  \\ \hline
  PDT & predeterminer & both his children  \\ \hline
  POS & possessive ending & 's \\ \hline
  PRP & pronoun, personal & me, you, it  \\ \hline
  PRP\$ & pronoun, possessive & my, your, our  \\ \hline
  RB & adverb & extremely, loudly, hard   \\ \hline
  RBR & adverb, comparative & better  \\ \hline
  RBS & adverb, superlative & best  \\ \hline
  RP & adverb, particle & about, off, up  \\ \hline
  SYM & symbol & \%  \\ \hline
  TO & infinitival to & what to do?  \\ \hline
  UH & interjection & oh, oops, gosh  \\ \hline
  VB & verb, base form & think  \\ \hline
  VBZ & verb, 3rd person singular present & she thinks  \\ \hline
  VBP & verb, non-3rd person singular present & I think  \\ \hline
  VBD & verb, past tense & they thought  \\ \hline
  VBN & verb, past participle & a sunken ship  \\ \hline
  VBG & verb, gerund or present participle & thinking is fun  \\ \hline
  WDT & wh-determiner & which, whatever, whichever  \\ \hline
  WP & wh-pronoun, personal & what, who, whom  \\ \hline
  WP\$ & wh-pronoun, possessive & whose, whosever  \\ \hline
  WRB & wh-adverb & where, when  \\ \hline
  . & punctuation mark, sentence closer & .;?*  \\ \hline
  , & punctuation mark, comma & ,  \\ \hline
  : & punctuation mark, colon & :  \\ \hline
  ( & contextual separator, left paren & (  \\ \hline
  ) & contextual separator, right paren & ) \\ \hline
  \caption{Tagset Penn Treebank} \label{tab:penn-tagset}
\end{longtable}

\subsection{Evalita 2009}
\emph{Evalita} nasce grazie all'iniziativa dell'\emph{Italian Association for Computational Linguistics} (ALIC),
ed \`e stato approvato dall'\emph{Italian Association for Artificial Intelligence} (AI*IA)
e dall'\emph{Italian Association for Speech Science} (AISV).
Lo scopo del progetto \`e quello di promuovere lo sviluppo di tecnologie in ambito linguistico,
scritto e parlato, per la lingua italiana, fornendo un ambiente condiviso nel quale
differenti sistemi e approcci, possono essere sviluppati e valutati in maniera consistente.

La diffusione di attivit\`a e di metodologie di valutazione condivise costituisce un passo fondamentale
verso lo sviluppo di risorse e tecnologie di Natural Language Processing. Il buon riscontro
ottenuto da Evalita, sia in termini di partecipanti che in termini di qualit\`a dei risultati ottenuti,
ha dimostranto che vale la pena perseguire tali obiettivi anche per la lingua italiana.
Inoltre, i dati di test e di training per le attivit\`a proposte, vengono resi disponibili alla
comunit\`a scientifica come punto di riferimenti per futuri miglioramenti.

In particolare, per questa tesi, si far\`a riferimenti all'attivit\`a di
PoS-Tagging proposta da Evalita nel 2009.

I dataset forniti dagli organizzatori sono costituiti da articoli tratti dall'edizione
online del giornale \emph{La Repubblica} (http://www.repubblica.it).

L'intero corpus \`e formato da 108,874 parole divise in 3,719 frasi.

Questo \`e stato annotato in pi\`u passaggi: il primo \`e stato portato a termina dal
gruppo di Andrea Baroni, dell'Universit\`a di Bologna, che ha classificato manualmente
l'intero corpus adottando un modello di classificazione con poche classi; successivamente
\`e stato utilizzato \emph{MorphIt!}, uno strumento automatizzato, per assegnare una lista di
possibili classi morfologiche a ciascuna parola; il risultato \`e stato poi convertito, per mezzo
di uno script, nel modello di classificazione \emph{Tanl}.

Infine, l'intero corpus \`e stato revisionato manualmente.

\subsubsection{Formato dei dati}
Il dataset di training \`e formato da un unico file di testo, con codifica UTF-8,
dove ogni riga costituisce un token seguito dalla sua classe, separati da una tabulazione,
secondo il seguente schema:

\begin{center}
  \begin{minipage}{5cm}
    \begin{verbatim}
    <TOKEN_1> <TAG1>
    <TOKEN_2> <TAG2>
    ...
    <TOKEN_N> <TAGN>
    <RIGA VUOTA>
    \end{verbatim}
  \end{minipage}
\end{center}

Al termine di ogni frase \`e presenta una riga vuota. Ad esempio:

\begin{center}
  \begin{minipage}{5cm}
    \begin{verbatim}
      A             E
      ben           B
      pensarci      Vfc
      ,             FF
      l'            RDns
      intervista    Sfs
      dell'         EAns
      on.           SA
      Formica       SP
      e'            VAip3s
      stata         VApsfs
      accolta       Vpsfs
      in            E
      genere        Sms
      con           E
      disinteresse  Sms
      .             FS

    \end{verbatim}
  \end{minipage}
\end{center}

Nell'esempio precedente vengono mostrati alcuni fra i pi\`u comuni problemi di tokenizzazione:
\begin{itemize}
  \item Le abbreviazioni vengono trattate come token (\emph{on.})
  \item Possibili espressioni multi parola non vengono trattate come un unico token (\emph{in\_genere})
  \item I clitici non vengono separati dal token (\emph{pensarci})
\end{itemize}

\subsubsection{Il modello di classificazione Tanl}
Il modello \emph{Tanl} in base alle linee guida di \emph{EAGLES}, uno standard
comunemente accettato dalla comunit\`a NLP, ed \`e derivato dalla classificazione
morfologica adottata dal corpus ISST.

Tanl \`e composto da tre livelli di classi morfologiche, ciascun livello aggiunge maggior
dettaglio alla classificazione.

Il primo livello \`e composto da 14 classi (Tabella~\ref{tab:tanl-coarse}):

\begin{table}[!htbp]
  \centering
  \begin{tabular}{| c || l |}
    \hline
    \thead{Tag} & \thead{Descrizione} \\
    \hline
    A & adjective \\
    B & adverb \\
    C & conjunction \\
    D & determiner \\
    E & preposition \\
    F & punctuation \\
    I & interjection \\
    N & numeral \\
    P & pronoun \\
    R & article \\
    S & noun \\
    T & predeterminer \\
    V & verb \\
    X & residual class \\ \hline
  \end{tabular}
  \caption{Tagset Tanl, primo livello} \label{tab:tanl-coarse}
\end{table}

Il secondo livello del modello Tanl contiene 36 classi, di seguito riportate, con relativi esempi (Tabella~\ref{tab:tanl-fine}):
\begin{longtable}{| c | p{.20\textwidth} | p{.30\textwidth} | p{.40\textwidth} |} \hline
  \thead{Tag} & \thead{Descrizione} & \thead{Esempio} & \thead{Contesto} \\ \hline
  A & adjective & bello, buono, pauroso, ottimo  & \parbox[t]{.40\textwidth}{una \emph{bella} passeggiata\\un \emph{ottimo} attaccante\\una persona \emph{paurosa}}\\ \hline
  AP & possessive adjective & mio, tuo, nostro, loro & \parbox[t]{.40\textwidth}{a \emph{mio} parere\\il \emph{tuo} libro}\\ \hline
  B & adverb & bene, fortemente, malissimo, & \parbox[t]{.40\textwidth}{arrivo \emph{domani}\\sto \emph{bene}}\\ \hline
  BN  & negation adverb & non & \emph{non} sto bene\\ \hline
  CC & coordinative conjunction & e, o, ma & \parbox[t]{.40\textwidth}{i libri \emph{e} i quaderni\\vengo \emph{ma} non rimango}\\ \hline
  CS & subordinative conjunction & mentre, quando & \parbox[t]{.40\textwidth}{\emph{quando} ho finito vengo\\\emph{mentre} scrivevo ho finito l'inchiostro}\\ \hline
  DD  & demonstrative determiner & questo, codesto, quello & \parbox[t]{.40\textwidth}{\emph{questo} denaro\\\emph{quella} famiglia}\\ \hline
  DE & exclamative determiner & che, quale, quanto & \parbox[t]{.40\textwidth}{\emph{che} disastro!\\\emph{quale} catastrofe!}\\ \hline
  DI & indefinite determiner & alcuno, certo, tale, parecchio, qualsiasi & \parbox[t]{.40\textwidth}{\emph{alcune} telefonate\\\emph{parecchi} giornali\\\emph{qualsiasi} persona}\\ \hline
  DQ & interrogative determiner & cui, quale & i \emph{cui} libri\\ \hline
  DR & relative determiner & che, quale, quanto & \parbox[t]{.40\textwidth}{\emph{che} cosa\\\emph{quanta} strada\\\emph{quale} formazione}\\ \hline
  E & preposition & di, a, da, in, su, attraverso, verso, prima\_di & \parbox[t]{.40\textwidth}{\emph{a} casa\\\emph{prima\_di} giorno\\\emph{verso} sera}\\ \hline
  EA & articulated preposition & alla, del, nei & \emph{nel} posto\\ \hline
  FB & balanced punctuation & ( ) [ ] { } - \_ ` & \emph{(}sempre\emph{)}\\ \hline
  FF & comma, hyphen & , - & carta, penna, 30\emph{-}40 persone\\ \hline
  FS & sentence boundary punctuation & . ? ! ... & cosa\emph{?}\\ \hline
  I & interjection & ahim\`e, beh, ecco, grazie & \emph{Beh}, che vuoi?\\ \hline
  N & cardinal number & uno, due, cento, mille, 28, 2000 & \parbox[t]{.40\textwidth}{\emph{due} partite\\\emph{28} anni}\\ \hline
  NO & ordinal number & primo, secondo, centesimo & \emph{secondo} posto\\ \hline
  PC & clitic pronoun &mi, ti, ci, si, te, ne, lo, la, gli & \parbox[t]{.40\textwidth}{me \emph{ne} vado\\\emph{si} sono rotti\\\emph{mi} lavo\\\emph{gli} parlo}\\ \hline
  PD & demonstrative pronoun & questo, quello, costui, ci\`o & \parbox[t]{.40\textwidth}{\emph{quello} di Roma\\\emph{costui} uccide}\\ \hline
  PE & personal pronoun & io, tu, egli, noi, voi & \parbox[t]{.40\textwidth}{\emph{io} parto\\\emph{noi} scriviamo}\\ \hline
  PI & indefinite pronoun & chiunque, ognuno, molto & \parbox[t]{.40\textwidth}{\emph{chiunque} venga\\i diritti di \emph{ognuno}}\\ \hline
  PP & possessive pronoun & mio, tuo, suo, loro, proprio & \parbox[t]{.40\textwidth}{il \emph{mio} \`e qui\\pi\`u bella della \emph{loro}}\\ \hline
  PQ & interrogative pronoun & che, chi, quanto & \parbox[t]{.40\textwidth}{non so \emph{chi} parta\\\emph{quanto} costa?\\\emph{che} ha fatto ieri?}\\ \hline
  PR & relative pronoun & che, cui, quale ci\`o & \parbox[t]{.40\textwidth}{\emph{che} dice\\il \emph{quale} afferma\\a \emph{cui} parlo}\\ \hline
  RD & determinative article & il, lo, la, i, gli, le & \parbox[t]{.40\textwidth}{\emph{il} libro\\\emph{i} gatti}\\ \hline
  RI & indeterminative article & uno, un, una & \parbox[t]{.40\textwidth}{\emph{un} amico\\\emph{una} bambina}\\ \hline
  S & common noun & amico, insegnante, verit\`a & \parbox[t]{.40\textwidth}{l'\emph{amico}\\la \emph{verit\`a}}\\ \hline
  SA & abbreviation & ndr, a.C., d.o.c., km & \parbox[t]{.40\textwidth}{30 \emph{km}\\sesto secolo \emph{a.C.}}\\ \hline
  SP & proper noun & Monica, Pisa, Fiat, Sardegna & \emph{Monica} scrive\\ \hline
  T & predeterminer & tutto, entrambi & \parbox[t]{.40\textwidth}{\emph{tutto} il giorno\\\emph{entrambi} i bambini}\\ \hline
  V & main verb & mangio, passato, camminando & \parbox[t]{.40\textwidth}{\emph{mangio} la sera\\il peggio \`e \emph{passato}\\ho \emph{scritto} una lettera}\\ \hline
  VA & auxiliary verb & avere, essere, venire & \parbox[t]{.40\textwidth}{il peggio \`e \emph{passato}\\ho \emph{scritto} una lettera\\viene \emph{fatto} domani}\\ \hline
  VM & modal verb & volere, potere, dovere, solere & \parbox[t]{.40\textwidth}{non posso \emph{venire}\\vuole \emph{comprare} il libro}\\ \hline
  X & residual class  & it includes formulae, unclassified words, alphabetic symbols and the like & \parbox[t]{.40\textwidth}{distanziare di \emph{43''}\\mi \emph{piacce}}\\ \hline
  \caption{Tagset Tanl, secondo livello} \label{tab:tanl-fine}
\end{longtable}

Nella forma pi\`u completa Tanl conta 328 classi che includono informazioni morfologiche, codificate in questo modo:
\begin{itemize}
  \item \emph{genere}: m (maschile), f (femminile), n (non specificato)
  \item \emph{numero}: s (singolare), p (plurale), n (non specificato)
  \item \emph{persona}: 1 (prima), 2 (seconda), 3 (terza)
  \item \emph{modo}: i (indicativo), m (imperativo), c (congiuntivo), d (condizionale), g (gerundio), f (infinito), p (participio)
  \item \emph{tempo}: p (presente), i (imperfetto), s (passato), f (futuro)
  \item \emph{clitico}: c segnala la presenza di clitici aggiuntivi
\end{itemize}
