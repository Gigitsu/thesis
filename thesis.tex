%
% ----------------------------------------------------------------
% *************************PhD Thesis ************************
% ----------------------------------------------------------------
%\documentclass[twoside,openright,a4paper,fleqn]{book}
%\usepackage{a4}
\documentclass[final,a4paper]{book}
\usepackage[T1]{fontenc}
\usepackage[ansinew]{inputenc}
\usepackage{latexsym,amssymb}
\usepackage{makeidx}
\usepackage[chapter]{tocbibind}
\renewcommand{\figurename}{\footnotesize\sc Figure\rm}
\renewcommand{\tablename}{\footnotesize\sc Table\rm}

\usepackage[active]{srcltx}
\usepackage{fancyhdr}
\usepackage{graphicx}
\usepackage{epsfig}
\usepackage{algorithm}
\usepackage{algpseudocode}

\usepackage[italian]{babel}

%\usepackage[refpages]{gloss}

%\usepackage{subfigure} % was: subfignarms.sty
%\usepackage{psfig}


\usepackage{layout}
%Theorem
\newtheorem{theorem}{Theorem}
\newtheorem{lemma}{Lemma}
\newtheorem{definition}{Definition}
\newtheorem{notation}{Notation}

% ----------------------------------------------------------------
\vfuzz2pt % Don't report over-full v-boxes if over-edge is small
\hfuzz2pt % Don't report over-full h-boxes if over-edge is small

% ENVIRONMENTS --------------------------------------------------
\newlength{\larghezza}
\newlength{\inizio}


% ---------- intro -----------------------------------------------

\makeindex

% ---------- Glossary -------------------------------------
%\makegloss

% ---------- End Glossary ---------------------------------------

\linespread {1.3}

\oddsidemargin=54pt
\evensidemargin=54pt

\begin{document}
%\layout

\pagestyle{fancy}
\fancyhf{}
\fancyhead[RL]{\rightmark}
\fancyhead[LO]{\leftmark}
\fancyhead[LE,RO]{\small \thepage}

\renewcommand{\chaptermark}[1]{\markboth{\sc\small #1}{}}
\renewcommand{\sectionmark}[1]{\markright{\sc\small \thesection\ #1}}

%\pagenumbering{Roman}

%Titolo da rivedere e confermare
\title{\sc \LARGE Classification and Prediction on sequential data using Recurrent Neural Networks}
\author{\LARGE \textit{Luigi Clemente}\\
Dipartimento di Informatica\\
Dottorato in Informatica XXIII ciclo\\
\sc Universit\`a degli Studi di Bari ``Aldo Moro''\\
\sl Via E. Orabona, 4 - 70125 Bari, \sc Italy\\
\tt luigi.clemente@gsquare.it
}
\date{
%\begin{figure}[h]
 % \centering
  %\includegraphics[width=3cm]{fig/eu-flag.png}
%\end{figure}
\vspace{1cm}
S.S.D.: INF/01\\
\vspace{2cm}
Supervisore: Prof. Donato Malerba\\
\vspace*{\fill}
\begin{tabular}{c}
	\hline
	\it A dissertation submitted in partial fulfillment \\
	\it of the requirements for the degree of\\
	{\sc Doctor of Philosophy in Computer Science}\\
	\hline	\\
	16 Ottobre 2015
\end{tabular}
}

% ----------------------------------------------------------------

\pagenumbering{roman}

\maketitle

\clearpage

\section*{Crediti}
Questa tesi \`e stata impaginata usando i seguenti programmi open-source:
\begin{itemize}
  \item [$\circ$] TeXShop LaTeX Editor \\
	available at: {\tt http://www.uoregon.edu/$\sim$koch/texshop/}
  \item [$\circ$] MacTex Distribution \\
	available at: {\tt http://www.tug.org/mactex/}
\end{itemize}
\vspace*{\fill}


\noindent
\begin{tabular}{p{185pt} p{110pt}}
Supervisore Tesi \hskip 3cm \ & \\ \cline{2-2} & Prof. Donato Malerba \\
\end{tabular}

\vspace{2cm}

\noindent
\begin{tabular}{p{185pt} p{110pt}}
Presidente del Comitato di Supervisione \hskip 2cm \ & \\ \cline{2-2}
\end{tabular}
\vspace{1cm}

\noindent
\begin{tabular}{p{185pt} p{110pt}}
Membro  del Comitato di Supervisione \hskip 2cm \ & \\ \cline{2-2}
\end{tabular}
\vspace{1cm}

\noindent
\begin{tabular}{p{185pt} p{110pt}}
Membro  del Comitato di Supervisione  \hskip 2cm \ & \\ \cline{2-2}
\end{tabular}
\vspace{2cm}


\vspace*{\fill} \noindent \hrule
\begin{tabular}{l}
	Presentato il {\sl XXXX Date to define}\\
	Copyright {\copyright} 2015 Luigi Clemente\\
\end{tabular}
\hrule


% ----------------------------------------------------------------
% insert toc
\tableofcontents
% ----------------------------------------------------------------
% insert Lists
\listoffigures
\listoftables
% ----------------------------------------------------------------
% insert Acknowledgments

\chapter*{Riconoscimenti}
\addcontentsline{toc}{chapter}{Riconoscimenti}
\pagenumbering{arabic}
\markboth{\sc Riconoscimenti}{\sc Riconoscimenti}

\input{ack}
% ----------------------------------------------------------------
% insert Abstract
\chapter*{Sommario}
\addcontentsline{toc}{chapter}{Sommario}
\markboth{\sc Sommario}{\sc Sommario}
Innumerevoli compiti richiedono la consapevolezza del tempo. Ad esempio l'assegnare
didascalia ad immagini, effettuare una sintesi vocale o giocare ad un video gioco
sono tutti compiti che richiedono un modello in grado di generare delle sequenze
di output. In altri ambiti, come la predizione ti time series o l'analisi video
sono richiesti modelli capaci di apprendere da sequenze di input. Inoltre compiti
di gran lunga pi\`u interattivi come la traduzione di testi in linguaggio naturale
richiedono modelli che apprendano da sequenze di input e generino sequenze di output.

Le Reti Neurali Ricorrenti (o Recurrent Neural Networks RNNs) costituiscono un
sottoinsieme di reti neurali in grado di catturare le dinamiche temporali attraverso
l'uso di cicli nel grafo delle connessioni. A differenza delle reti neurali
tradizionali, le reti ricorrenti possono analizzare i campioni uno alla volta
mantenendo uno stato, o memoria, che riflette una finestra contestuale arbitrariamente
lunga. Mentre queste reti sono state a lungo ritenute troppo difficili da addestrare
dato che spesso contengono milioni di parametri, i recenti progressi nelle
architetture di rete, nelle tecniche di ottimizzazione e nella computazione
parallela hanno reso possibile l'apprendimento su larga scala con le RNNs.

Negli ultimi anni sistemi basati su architetture di reti ricorrenti come Long
Short Term Memory (LSTM) hanno dimostrato delle performance da record in vari
compiti come l'image captioning, traduzioni linguistiche e riconoscimento della
scrittura.

In questa tesi utilizzeremo una LSTM per creare un Part-Of-Speech (POS) Tagger.

In linguistica il pos tagging \`e un processo che assegna ad ogni parola di un
testo una particolare parte del discorso (es. sostantivo, aggettivo, pronome ecc..).
Questa assegnazione viene fatta basandosi sia sulla parola stessa che sul contesto.
La quasi totalit\`a di algoritmi di PoS-Tagging oggi esistente pu\`o essere divisa
n due categorie:

\begin{itemize}
  \item basati su regole
  \item probabilistici
\end{itemize}

Gli algoritmi basati su regole contano un gran numero di regole scritte a mano,
basate sulle caratteristiche morfologiche del linguaggio. Ad esempio il fatto che
``\emph{quickly}'' finisca in ``\emph{ly}'' o che ``\emph{shopping}'' finisca in
``\emph{ing}'' dovrebbe aiutare ad identificare queste parole come verbi. \`E
evidente che queste regole sono fortemente dipendenti dal linguaggio per il quale
sono state scritte. L'esempio precedente vale per l'inglese ma non per l'italiano.
Algoritmi di questo tipo, pertanto, dovranno essere scritti manualmente per ogni
linguaggio conosciuto dall'uomo.

Gli algoritmi probabilistici, invece, utilizzano un corpus taggato da usare per
addestrare un qualche tipo di modello. Questi prevedono sempre una fase di
preprocessing del testo.

Per essere accurati, infatti, il testo dev'essere \emph{tokenizzato}, ossia il
testo dev'essere diviso in parole. Questo procedimento non \`e banale come pu\`o
sembrare, non sempre ci si pu\`o limitare a dividere una frase in base agli spazi.
Ad esempio, la forma pi\`u comune di cinese continentale non ha spazi bianchi e
non \`e segmentato in alcun modo. Ma ci sono problemi anche con un linguaggio
morfologicamente non troppo ricco come l'inglese. A titolo esemplificativo,
tentiamo di dividere (tokenizzare) \texttt{aren't}. Abbiamo pi\`u d'una possibilit\`a,
\texttt{aren't}, \texttt{arent}, \texttt{are nt} e \texttt{aren t} sono tutte
ugualmente valide. Altri linguaggio propongono sfide pi\`u grandi, come le parole
composte (\emph{Komposita}) del tedesco. Gli algoritmi di tokenizzazione esistenti
sono per lo pi\`u basati su regole e, per tanto, anche questi dipendenti dal
linguaggio per il quale sono stati scritti.

Altri algoritmi di uso comune durante la fase di preprocessing sono gli algoritmi
di \emph{stemming} o \emph{lemmatizzazione}. In ogni lingua la flessione \`e il
modo in cui varia morfologicamente a partire da una radice linguistica comune,
per indicarne tratti grammaticali o sintattiche. Gli algoritmi di lemmatizzazione
riportano la forma flessa di ogni parola sotto un unico \emph{lemma}. Anche per
quanto riguarda questi algoritmi, la maggior parte delle implementazioni \`e
basata su regole create ad hoc per una determinata lingua

Lo scopo di questa tesi \`e quello di provare a superare queste limitazioni,
addestrando un PoS-Tagger che classifichi il testo a partire dai singoli caratteri,
piuttosto che le singole parole. Il classificatore verr\`a addestrato a partire
da un corpus precedentemente taggato. In questo modo sar\`a evitata tutta la parte
di preprocessing e tokenizzazione del testo, rendendo il classificatore indipendente
dal linguaggio utilizzato

% ----------------------------------------------------------------
\clearpage
% ----------------------------------------------------------------


\chapter {Introduzione}\label{chap:chap_1}
\nocite{Lipton:2009}
Fin dagli albori dell'informatica l'uomo \`e sempre stato affascinato dalla possibilit\`a di creare un'intelligenza artificiale - macchine in grado di riprodurre la complessit\`a dei comportamenti umani.
Alan Turing, uno dei padri dell'informatica moderna, ha dimostrato come qualsiasi problema computabile pu\`o essere eseguito da una Macchina di Touring Universale~\cite{wiki:MacchinaTuring} - quindi, se la mente umana pu\`o essere descritta da un algoritmo allora una qualsiasi Macchina di Turing \`e abbastanza potente da rappresentarla.

Tutti i computer odierni sono Turing Completi, il che significa che possono rappresentare ogni tipo di algoritmo computabile.
Il problema quindi \`e trovare il giusto algoritmo capace di simulare il comportamento umano; tuttavia, questo risulta troppo complesso per essere risolto nell'immediato ma possiamo pensarea a diversi approcci - ad esempio - possiamo partire creando macchine in grado di riconoscere semplici immagini come numeri scritti, migliorarle rendendole capaci di riconoscere volti umani e proseguendo cos\`i fino alla creazione di macchine con la stessa abilit\`a umana di riconoscere oggetti nel mondo reale.

Un approccio alternativo consiste nel simulare il cervello umano a livello di neuroni.
Con la tecnologia attuale possiamo simulare la realt\`a in modo molto realistico, basti pensare ai progressi fatti in campo videoludico o cinematografico; risulta perci\`o abbastanza realistico ipotizzare che, con la giusta rappresentazione di un neurone e sufficiente capacit\`a di calcolo, sia possibile simulare un cervello umano e creare macchine intelligienti.

Ed \`e proprio da questa intuizione che nasce il concetto di reti neurali artificiali.

\section{Reti Neurali}
Gli studi sul sistmea nervoso umano hanno ispirato il concetto di rete neurale artificiale.
In una rete neurale artificiale, dei semplici nodi artificiali (detti "neuroni" o "unit\`a") sono collegati gli uni agli altri in modo da formare una rete che imitia una rete neurale biologica.

Non esiste un'unica definizione formale di cosa \`e una rete neurale.
Comunque, una classe di modelli statistici possono essere chiamati "neurali" se hanno le seguenti caratteristiche:

\begin{itemize}
  \item contengono un insieme di pesi, ad esempio dei parametri numerici che vengono regolati tramite un algoritmo di apprendimento
  \item sono capaci di approssimare una funzione non lineare degli input
\end{itemize}

Si pu\`o pensare a questi pesi come la forza con cui i singoli neuroni sono collegati gli uni con gli altri, collegamenti che vengono attivati durante le fasi di addestramento e predizione.

Nelle attuali implementazioni software delle reti neurali, l'approccio ispirato alla biologia \`e stato per lo pi\`u abbandonato in favore di un'approccio pi\`u pratico basado sulla statistica e sulla teoria dei segnali.

La prima e pi\`u semplice tipologia di rete neurale ideata \`e stata la \emph{feedforward neural network}.
In queste reti le infomrazioni viaggiano in una sola direzione, in avanti (forward) dall'input verso l'output attraversando eventuali nodi nascosti.
Le connessioni tra le unit\`a della rete non formano cicli diretti (Figura~\ref{fig:feedforwardNeuralNetwork}).

\section{Il Tempo} %TODO cambiare titolo?
Un grosso limite delle reti neurali feedforward \`e l'incapacit\`a di modellare esplicitamente il tempo.
Tutta via le reti feedforward, cos\`i come altri modelli per l'apprendimento automatizzato (es. support vector machine), si sono dimostrate estremamente capaci anche senza una modellazione esplicita del tempo.
Anzi, probabilmente \`e proprio grazie a questa assunzione di indipendenza fra i campioni di una sequenza di input che negli ultimi anni siamo stati in grado di avanzare cos\`i tanto nel campo del machine learning.
In pi\`u molti modelli cercando di catturare il tempo concatenando i singoli input o con un valore numerico, a rappresentare l'ordine temporale, o con i suoi diretti successori o predecessori nella sequenza degli input fornendo cos\`i una sorta di contesto.

Sfortunatamente, nonostante l'utilit\`a dell'assunzione di indipendenza, questa rende impossibile modellare delle dipendenze temporali a lungo raggio.
Ad esempio, un modello addestrato con una finestra contestuale di lunghezza finita pari a 5 non sar\`a mai in grado di rispondere a semplici domande del tipo ``\emph{qual \`e stata l'informazione vista 10 unit\`a temporali fa?}''.
Per applicazioni pratiche, come ad esempio l'automatizzazione di un call center, un sistema limitato come questo potrebbe riuscire ad indirizzare correttamente le chiamate ma non potrebbe mai partecipare ad una coversazione.

Fin dai primi modelli di intelligenza artificiale, abbiamo cercato di creare sistemi capaci di interagire con gli esserei umani.
Alan Turing, nel suo saggio \emph{Computing Machinery and Intelligence}, propose un ``Gioco dell'Imitazione'' con il quale giudicare l'intelligenza di una macchina in base alla sua abilit\`a di sostenere un dialogo convincente con un essere umano~\cite{Turing:1950}.
Oltre a sistemi in grado di sostenere un dialogo convincente, ne esistono molti altri la cui realizzazione sarebbe altrettanto utile ed importante (es. automobili capaci di spostarsi in autonomia e sistemi di chirurgia robotica) e sembra molto improbabile riuscire a raggiungere certi risultati senza utlizzare una modellazione esplicita del tempo.

Le \emph{reti neurali ricorrenti} (\emph{RNN}) sono state pensate proprio per superare questi limiti.
Si tratta di un sovrainsieme delle reti neurali feedforward con la capacit\`a di passare informazioni attraverso fasi temporali e, per questo, in grado di eseguire quasi ogni tipo di computazione.

Un noto risultato di Siegelman and Sontag del 1991 dimostra come una rete neurale ricorrente di dimensione finita e con delle funzioni sigmoide come funzioni di attivazione possa simulare una macchina di Turing universale~\cite{Siegelmann:1991}.
In pratica, l'abilit\`a di modellare dipendenze temporali rende le reti neurali particolarmente adatta a compiti dove i dati in input e/o output sono costutiti da sequeze di valori che sono dipendenti gli uni dagli altri (Figura~\ref{fig:simpleRecurrentNeuralNetwork}).

\section{Modelli di Markov} %TODO pensare ad un titolo piu' appropiato
Le reti neurali ricorrenti per\`o non sono stati i primi modelli a catturare le dipendenze temporali.
Le \emph{catene di Markov}, che modellano le transizioni tra sequenze di stati $(s^{(1)}, s^{(2)}, \dots, s^{(T)},)$, sono state inizialmente descritte dal matematico Andrey Markov nel 1906.
I \emph{modelli di Markov nascosti} (\emph{HMM}), che modellano i dati osservati $(o^{(1)}, o^{(2)}, \dots, o^{(T)},)$ come probabilisticamente dipendenti da stati non osservati, sono stati descritti nel 1950 e sono stati largamente studiati fino agli anni '60.
In ogni caso, l'approccio tradizionale con il modello di Markov \`e piuttosto limitato perch\`e i possibili stati possono essere presi solo da un piccolo spazio discreto di stati $ s_j \in S$.
L'algoritmo Viterbi, usato in fase di apprendimento, ha una complessit\`a in tempo pari ad $O(|S|^2)$.
In pi\`u la matrice di transizione, che cattura la probabilit\`a di muoversi da uno stato in uno dei possibili stati adiacenti, ha dimensione $|S|^2$.
Quindi, risulta impossibile utilizzare un HMM quando l'insieme dei possibili stati nascosti \`e pi\`u grande di circa 106.
Inoltre, ciascun stato nascosto $s^{(t)}$ pu\`o dipendere solo dallo stato precedente $s^{(t+1)}$
E nonostante sia possibile estendere il modello di Markov in modo da gestire pi\`u di un solo stato precedente aumentando cos\`i la finestra contestuale usata per l'apprendimento, questo porterebbe anche ad una crescita esponenziale dello spazio degli stati, rendendo il modello di Markov computazionalmente impraticabile per modelli con dipendenze a lungo raggio.

Queste limitazioni invece non si pongono per le reti neurali ricorrenti.
Le reti neurali ricorrenti, infatti, possono catturare dipendenze temporali a lungo raggio, superando le limitazioni imposte dal modello di Markov.
Cos\`i come nei modelli di Markov, ogni stato in una RNN tradizionale dipende solo dai valori di input correnti e dallo stato della rete nella precedente fase temporale\footnote{Le \emph{reti neurali ricorrenti bidirezionali} (\emph{BRNN})~\cite{Schuster:1997} estendono le RNN per modellare dipendenze con osservazioni passate e future mentre le tradizionali RNN modellano dipendenze solo con osservazioni passate.}.
Tuttavia, gli stati nascosti possono contenere, ad ogni fase temporale, informazioni provenienta da una finestra contestuale arbitrariamente lunga.
Questo \`e possibile grazie al fatto che il numero di stati distinti che possono essere rappresentati da un livello nascosto di nodi cresce in maniera esponenziale all'aumentare del numero dei nodi presenti nel livello.
Anche se ciascun nodo pu\`o rappresentare solo valor binari, un singolo livello nascosto pu\`o rappresentare $2^N$ stati distinti dove $n$ \`e il numero di nodi presenti.
Con dei valori reali invece, anche tenendo conto del limite imposta dalla rappresentazione a 64 bit dei numeri, un singolo livello nascosto pu\`o rappresentare $2^64N$ stati distinti.
Nonostante il potenziale potere espressivo della rete cresce in maniera esponenziale, la complessit\`a di addestramento e predizione cresce in ordine quadratico.

\mytodo{aggiungere altro?} %TODO aggiungere altro?


\chapter {Stato dell'arte}\label{chap:chap_2}
In questo capitolo introduciamo notazioni formali e diamo una breve spiegazione del funzionamento delle reti neurali e degli strumenti utilizzati.

\section{Il Tempo}

Le RNN non sono limitate a sequenze indicizzate in maniera temporale.
Sono state usate con successo anche per seqenze di dati non temporali, come ad esempio i dati genetici.
In ongi caso, la computazione procede nel tempo e molte importanti applicazioni hanno un aspetto temporale esplicito o implicito.

Nonostante, in questa tesi, ci riferiremo al tempo i metodi descritti sono applicabili ad una famiglia pi\`u ampia di compiti.
Parlando di tempo ci riferiamo ad un campione $x^{(t)}$ in input e ad un valore atteso $y^{(t)}$ in output che sono generati in \emph{sequenze di fasi temporali} discrete indicate da $t$
La nostra sequenza pu\`o essere formata da un numero finito di campioni o da un numero infinito ma numerabile di campioni.
Se abbiamo a che fare con un numero finito di campioni allora possiamo indicare con $T$ il massimo indice temporale.
Quindi una sequenza di valori di input consecutivi pu\`o essere scritta come $(x^{(1)}, x^{(2)}, \dots, x^{(T)})$ mentre gli output come $(y^{(1)}, y^{(2)}, \dots, y^{(T)})$.
Questi valori possono essere dei campioni, presi ad intervalli regolari di tempo, di un processo reale continuo come ad esempio i fotogrammi che compongono un video.
Gli intervalli di tempo possono anche essere dei semplici valori ordinali senza una durata esatta.
\`E il caso, ad esempio, delle sequenze genetiche che hanno un ordine ma non un ordine temporale o ancora del linguaggio naturale dove le parole hanno un ordine logico ben preciso che, tuttavia, non corrisponde ad intervalli di temporali ben defniti.
Ad esempio nella frase ``\emph{Lisa suona il sassofono}'' abbiamo che $x^{(1)}$ = Lisa, $x^{(2)}$ = suona, ecc. Ciascuna parola corrisponde ad intervalli di tempo che non sono costanti, ``\emph{il}'' e ``\emph{sassofono}'' hanno bisogno di tempi diversi per essere pronunciati.

\section{Reti Neurali}
Le reti neurali sono modelli computazionali ispirati dalla biologia del sistema nervoso centrale.
Generalmente una rete neurale \`e formata da un insieme di \emph{neuroni artificiali}, comunemente chiamati \emph{nodi} o \emph{unit\`a}, collegati da un insieme di archi diretti che, intuitivamente, rappresentano le \emph{sinapsi} di una rete neurale biologica.
Associato ad ogni neurone $j$ vi \`e una funzione di attivazione $l_j$, chiamata anche funzione di collegamento (o funzione link).
In questa tesi user\`o la notazione ``$l_j$'' invece di ``$h_j$'' (notazione usata in altri documenti) per distinguere la funzione di attivazione $l_j$ dal valore dei nodi nascosti in una rete, che vengono comunemente indicati con \textbf{h} nella letteratura delle RNN.

Associato ad ogni arco dal nodo $j^{'}$ al nodo $j$ vi \`e un peso $w_{jj^{'}}$. Seguento la convenzione adottata in molti altri documenti che trattano reti neurali, indicheremo i neuroni con $j$ e $j^{'}$ mentre, con $w_{jj^{'}}$, indicheremo il peso corrispondente all'arco diretto che parte dal nodo $j^{'}$ e arriva al nodo $j$.
\`E importante notare che in altri documenti e libri, come ad esempio su Wikipedia, gli indici dei pesi sono invertiti e che $w_{j^{'}j} \neq w_{jj^{'}}$ indica il peso sull'arco diretto dal nodo $j^{'}$ al nodo $j$.

Il valore $v_j$ di ciascun neurone $j$ \`e calcolato applicando la sua funzione di attivazione ad una somma pesata dei suoi valori di input (Fugura~\ref{fig:artificialNeuron}): %TODO pensare ad un titolo piu' appropiato
\begin{equation} % \begin{equantion*} per non numerare l'equazione
  v_j = l_j\left( \sum_{j^{'}} w_{jj^{'}} \cdot v_{j^{'}} \right)
\end{equation}

\begin{figure}[tp]
  \centering
  \begin{center}
    \includegraphics[width=0.7\textwidth]{./images/artificialNeuron.png}
  \end{center}
  \caption{Un neurone artificiale calcona una funzione non lineare della somma pesata dei suoi input}
  \label{fig:artificialNeuron}
\end{figure}
Per comodit\`a, ci riferiremo alla somma pesata all'interno delle parentesi come l'\emph{attivazione in arrivo} e la indicheremo con $a_j$. Rappresentiamo l'intero processo in figura disegnando i neuromi coe dei cerchi mentre gli archi come delle frecce che li collegano.
Quando possibile, verr\`a utilizato un simbolo per indicare l'esatta funzione di attivazione utilizzata, ad esempio $\sigma$ per la fnzione sigmoide.

Scelte abbastanza comuni per la funzione di attivazione includono la funzione sigmoide $\sigma(z) = 1/(1+e^{-z})$ e la funzione \emph(tanh) $\phi(z)=(e^z-e^{-z})/(e^z+e^{-z})$ che \`e diventata molto comune nelle reti neurali di tipo \emph{feedforward} ma \`e stata utilizzata anche in reti neurali ricorrenti~\cite{Sutskever:2011}.
Un'altra funzione di attivazione che \`e diventata lo stato dell'arte nella ricerca di deep learning \`e la funzione ReLU (\emph{rectified linear unit}) $l_j(z)=\operatorname{max}(0, z)$.
Questa funzione ha dimostrato di poter migliorare le prestazioni di molte reti neurali in una grande variet\`a di applicazioni, che spaziano dal riconoscimento vocale al riconoscimento di oggetti, ed \`e stata utilizzata anche in reti neurali ricorrenti~\cite{Bengio:2013}.

La funzione di attivazione da applicare sui nodi di output dipende dall'applicazione.
Per una classificazione a pi\`u classi, applichiamo al livello di output una funzione non lineare softmax.
La funzione softmax calcola l'output come:
\begin{equation}
  \hat{y_k} = \frac{e^{a_k}}{\sum_{k^{'}=1}^{K} e^{a_{k^{'}}}}
\end{equation}
dove $K$ \`e il numero totale di possibili output (classi). Il denominatore \`e una funzione di normalizzazione che consiste nella somma di funzioni esponenziali dei valori dati in output da tutti i nodi e serve per assicrarsi che l'output totale sommi ad 1.
Nel caso di regressioni invece viene comunemente utilizzata una funzione lineare come output.
Dato che nella maggioranza dei casi le reti neurali, specialmente quelle ricorrenti, vengono utilizzate per applicazioni che coinvolgono la classificazione, durante questa tesi, a meno che diversamente specificato, daremo per scontato l'uso della funzioen softmax come output.

\section{Reti Neurali Feedforward}
Con un modello computazionale a rete neurali, \`e necessario determinare l'ordine con cui la computazione dovrebbe procedere.
I nodi dovrebbero essere calcolati uno alla volta e poi aggiornati, oppure i valori di tutti i nodi dovrebbero essere calcolati iniseme per poi applicare tutti gli aggiornamenti simultaneamente?
Le \emph{reti neurali feedforward} (Figura~\ref{fig:feedforwardNeuralNetwork}) sono una classe ristretta di reti neurali che affrontano questo prolema proibnedo i cicli dal grafo delle connessioni neurali.
In questo modo tutti i nodi possono essere disposti in livelli.
I valori in output di ciascun livello possono essere calcolati solo a partire dai valori di output dei livelli precedenti.
\begin{figure}[tp]
  \centering
  \begin{center}
    \includegraphics[width=0.7\textwidth]{./images/feedForwardNeuralNetwork.png}
  \end{center}
  \caption{Un modello di rete feedforward.
  Un campione \`e dato in pasto alla rete impostando i valori dei nodi blu.
  Ogni livello successivo \`e poi calcolato come una funzione dei livelli precedenti finch\`e non si raggiunge il livello pi\`u in alto.}
  \label{fig:feedforwardNeuralNetwork}
\end{figure}

L'input $x$ viene dato in pasto ad una rete neurale feedforward impostando i valori dei nodi del livello pi\`u in basso.
I valori dei nodi di ciascuno dei livelli superiori non potranno essere calcolati finch\`e non saranno disponibili i valori in output $\hat{y}$ dei livelli inferiori.
Queste tipologie di reti sono usate di frequente per applicazioni di apprendimento supervisionato come classificazione e regressione.
L'apprendimento \`e ottenuto aggiornando iterativamente i pesi dei singoli archi in modo da minimizzare una funzione di perdita, $\mathcal{L}(\hat{y},y)$, che penalizza la distanza fra l'output desiderato $y$ e l'output predetto $\hat{y}$ tramite tecniche di ottimizzazione.
Nonostante l'algoritmo di ottimizzazione esatto \`e un noto problema NP-Completo, una grande quantit\`a di euristiche pre addestramento e avanzate tecniche di ottimizzazione hanno condotto ad un impressionante numero di successi empirici su molte applicazioni di apprendimento supervisionato.

L'algoritmo utilizzato con maggiore successo per addestrare una rete neurale \`e l'algoritmo di backpropagation, introdotto da Rumelhart et al. nel 1985~\cite{Rumelhart:1985}.
Questo algoritmo usa la regola della catena per calcolare la derivata di una funzione di perdita $\mathcal{L}$ rispetto ciascun parametro nella rete.
I pesi sui singoli archi vengono poi tramite la discesa dei gradienti.
Dato che la superficie di perdita non \`e convessa non vi \`e alcuna garanzia che l'algoritmo di backpropagation riesca a trovare un minimo globale.
Ci\`o nonostante, nella pratica, reti addestrate in questo modo hanno ottenuto notevoli successi.

Tuttavia le reti feedforward sono limitate.
Dopo che ciascun campione \`e stato processato l'intero stato della rete viene perso.
Se i campioni sono indipendenti gli uni dagli altri questo non presenta assolutamente un problema.
Ma se i dati sono in una relazione temporale, questo non \`e accettabile.
I fotogrammi di un video o le parole di una frase rappresentano situazioni in cui l'assunzione di indipendenza fallisce.

\section{Reti Neurali Ricorrenti}
Le \emph{reti neurali ricorrenti} costituiscono un sovrainsieme proprio delle reti neurali feedforward che, a differenza di quest'ultime, includono degli archi ricorrenti.
Questi archi ricorrenti si estendono su intervalli temporali adiacenti e introducono il concetto di tempo nel modello.
Mentre le RNN possono non contenere cicli tra gli archi convenzionali, gli archi ricorrenti possono formare cicli.
Al tempo $t$, i nodi che ricevono un input da un arco ricorrente, ricevono un \emph{input di attivazione} sia dal campione corrente $x^{(t)}$ che dai nodi nascosti $h^{(t-1)}$ del precedente stato della rete.
L'output $\hat{y}^{(t)}$ \`e poi calcolato in base allo stato nascosto $h^{(t)}$ del tempo $t$.
Quindi, l'input $x^{(t)}$ al tempo $t-1$ pu\`o influenzare l'output $\hat{y}^{(t)}$ al tempo $t$ proprio grazie a queste connessioni ricorrenti.

Le seguenti due equazioni, mostrano i calcoli necessari per eseguire, per ogni fase temporale, il passaggio in avanti dei dati di una semplice rete neurale ricorrente:
\begin{equation}
  h^{(t)} = \sigma(W_{hx}x + W_{hh}h^{(t-1)} + b_h)
\end{equation}
\begin{equation}
  \hat{y}^{(t)} = \operatorname{softmax}(W_{yh}h^{(t)} + b_y)
\end{equation}

dove $W_{hx}$ \`e la matrice dei pesi tra il livello di input e il livelo nascosto mentre $W_{hh}$ \`e la matrice dei pesi ricorrenti fra i livelli nascosti di due fasi temporali adiacenti.
I vettori $b_h$ e $b_y$ rappresentano uno scostamento (\emph{bias}) che permettono a ciascun nodo di apprendere un offset.

I modelli discussi in questa tesi consistono in reti con livelli nascosti ricorrenti.
Tuttavia, sono stati proposti modelli, come la rete di Jordan, che ammettono la presenza di connessioni tra gli output della rete in uno stato e il livello nascosto della rete nello stato successivo.

\begin{figure}[tp]
  \centering
  \begin{center}
    \includegraphics[width=0.7\textwidth]{./images/simpleRecurrentNeuralNetwork.png}
  \end{center}
  \caption{Un seplice rete neuralie ricorrente.
  Ad ogni passo temporle $t$, l'attivazione viene passata lungo gli archi continui, cos\`i come accadrebbe per una normale rete feedforward.
  Gli archi tratteggiati collegano il nodo sorgente $j^{'}$ al tempo $t$, quindi $j^{'(t)}$, al nodo target del successivo passo temporale $j^{(t+1)}$.}
  \label{fig:simpleRecurrentNeuralNetwork}
\end{figure}

Una semplice rete neurale \`e mostrata in Figura~\ref{fig:simpleRecurrentNeuralNetwork}.
La dinamica di questa rete attravesso pi\`u fasi temporali pu\`o essere visualizzata \emph{dispiegando} la rete (Figura~\ref{fig:unfoldedSimpleRecurrentNeuralNetwork}).
Con questa visualizzazione, il modello pu\`o essere interpretata come una rete non ciclica, ma piuttosto come una rete con un livello per intervallo di tempo a dei pesi condivisi tra gli intervalli temporali.
Diventa quindi chiaro come una rete dispiegata in questo modo pu\`o essere addestrata attraverso pi\`u fasi temporali usando l'algoritmo di backpropagation.
Questo algoritmo viene chiamato \emph{backpropagation through time} (BPTT, backpropagation attraverso il tempo), ed \`e stata introdotta nel 1990~\cite{Werbos:1990}

\begin{figure}[tp]
  \centering
  \begin{center}
    \includegraphics[width=0.7\textwidth]{./images/unfoldedSimpleRecurrentNeuralNetwork.png}
  \end{center}
  \caption{Visualizzazione della rete \emph{dispiegata}.}
  \label{fig:unfoldedSimpleRecurrentNeuralNetwork}
\end{figure}

\subsection{Addestramento}
L'apprendimento con le reti neurali ricorrenti \`e stato a lungo visto come qualcosa di difficile.
Cos\`i come per tutte le reti neurali, l'ottimizzazione della funzione di perdita \`e un problema NP-Completo.
Ma l'apprendimento con le reti ricorrenti pu\`o essere reso ancora pi\`u complesso a causa della difficolt\`a nell'apprendere delle dipendenze a lungo raggio.
Il noto problema della \emph{scomparsa} ed \emph{esplosione} dei gradienti si verifica quando l'errore viene propagato attraverso molte fasi temporali.
\mytodo{dire di pi\`u a riguardo?} L'impatto dell'input al tempo $\mathcal{T}$ sull'output al tempo $t$ esploder\`a esponenzialmente oppure raggiunger\`a rapidamente zero al crescere di $\mathcal{T} - t$, a seconda se il peso $\abs{w_{jj}}>1$ oppure $\abs{w_{jj}}<1$ ma anche in base alla funzione di attivazione utilizzata %TODO dire di piu
(ad esempio con una funzione di attivazione $l_j = \sigma$ si verificher\`a maggiormente il problema della sparizione del gradeinte, viceversa con la funzione ReLU $\operatorname{max}(0, x)$ il gradiente esploder\`a).

Una possibile soluzione al problema consiste nell'usare una versione leggermente modificate dell'algoritmo BPTT che prende il nome di \emph{truncated backpropagation through time} (TBPTT)~\cite{Williams:1989}.
Con l'algoritmo TBPTT viene impostato un valore che indicia il numero massimo di temporali lungo le quali pu\`o essere propagato l'errore.
In questo modo si attenua il problema dell'esplosione del gradiente perdendo, tuttavia, la capacit\`a di apprendere dipendenze a lungo raggio.

Il problema dell'ottimizzazione rappresenta un fondamentale ostacolo che non pu\`o essere risolto semplicemente modificando l'architettura della rete.
\`E noto dal 1993 che ottimizzare una rete neurale di anche solo 3 livelli costituisce un problema NP-Completo.
Tuttavia, recenti studi sia teorici che empirici, suggeriscono che il problema non \`e cos\`i insormontabile nella pratica come si potrebbe pensare.

Inoltre implementazioni sempre pi\`u performanti ed migliorate euristiche per il calcolo dei gradienti hanno reso l'addestramento delle RNN fattibile.
Ad esempio, implementazioni degli algorotmi di forward a backward propagation che sfruttano la GPU, come Theano e Torch (Sezione~\ref{sec:torch}), hanno semplificato la realizzazione di veloci algoritmi di apprendimento.

\subsection{Architetture moderne}
Sono due le architetture RNN di maggior successo per l'apprendimento di dati sequenziali e risalgono entrambe al 1997.
La prima, \emph{Long Short-Term Memory}, ideata da Hochreiter e Schmidhuber, introduce il concetto di cella di memoria, un'unit\`a computazionale che rimpiazza il tradizionale neurone artificiale nei livelli nascosti della rete.
Con queste celle di memoria, la rete \`e in grado di superare le difficolt\`a incontrate dalle precedenti implementazioni durante la fase di apprendimento.
La seconda, \emph{Bidirectional Recurrent Neural Network}, ideata da Schuster e Paliwal, introduce invece l'architettura BRNN nella quale, per calcolare l'output del tempo $t$, vengono usate tanto le informazioni provenenti dal passato quanto quelle provenienti dal futuro.
Questo approccio \`e in contrasto con i sistemi precedenti, dove solo gli input provenienti da fasi temporali passate potevano influenzare gli output.
Fortunatamente, le due architetture non sono mutuamente esclusive e sono state combinate con successo.

Nel corso di questa tesi ci concentreremo sull'architettura \emph{Long Short-Term Memory}.

\section{Long Short-Term Memory (LSTM)}
Nel 1997, per superare il problema della scomparsa del gradiente, Hochreiter e Schmidhuber introdussero il modello LSTM.
Questo modello somiglia ad una rete neurale standard con un livello nascosto ricorrente, con l'unica differenza che i normali nodi di un livello nascosto (Figura~\ref{fig:artificialNeuron}) sono rimpiazzati da celle di memoria (Figura~\ref{fig:memoryCell}). %TODO immagini
La cella di memoria contiene un nodo sul quale insiste un un arco ricorrente connesso a se stesso con peso pari ad 1, assicurandosi, in questo modo, che il gradiente possa passare attraverso molte fasi temporali senza scomparire o esplodere.

\begin{figure}[tp]
  \centering
  \begin{center}
    \includegraphics[width=0.7\textwidth]{./images/memoryCell.png}
  \end{center}
  \caption{Una cella di memoria del modello LSTM cos\`i come descritta inizialmente da Hochreiter et al. in~\cite{Hochreiter:1997}.
  Il nodo con l'arco ricorrente rappresenta lo stato interno $s$.
  La linea diagonale sta ad indicare che si tratta di una funzione lineare, non \`e applicata nessuna funzione di attivazione.
  I nodi con il simbolo ``$\prod$'' danno in output il prodotto degli input.
  Le linee tratteggiate indicano archi ricorrenti mentre quelle rosa hanno un peso costante pari ad 1.}
  \label{fig:memoryCell}
\end{figure}

Per distinguere la presenza di una cella di memoria da un nodo normale, indicheremo queste celle con $c$.

Il termine ``Long Short-Term Memory'' deriva dalla seguente intuizione.
Le reti neurali ricorrenti pi\`u semplici hanno una \emph{memoria a lungo termine} (\emph{long term memory}) sotto forma di pesi.
I pesi, durante l'apprendimento, cambiano molto lentamente, codificando, in questo modo, la conoscenza.
Queste hanno anche una \emph{memoria a breve termine} (\emph{short term memory}) sotto forma di attivazioni effimere, che passano dall'output di ciascun nodo nei nodi successivi.
Il modello LSTM introduce una sorta di memoria intermedia tramite l'uso delle celle di memoria.
Una cella di memoria \`e una composizione di unit\`a pi\`u semplici con la nuova aggiunta di nodi moltiplicativi, indicati nel diagramma con il simbolo $\prod$.
Di seguito sono descritti tutti gli elementi che compongono una cella di memoria.
\begin{itemize}
  \item \emph{Internal State:} Il cuore di ogni cella di memoria \`e un nodo $s$ con una funzione di attivazione lineare.
  Dato che indichiamo con $c$ una cella di memoria, il suo stato interno sar\`a indicato con $c_s$.
  \item \emph{Constant Error Carousel:} Lo stato interno $c_s$ ha un arco connesso a se stesso (ricorrente) con peso pari ad 1.
  Quest'arco, chiamato \emph{constant error carousel}, si estende tra fasi temporali adiacenti con un peso costante, questo assicura che l'errore possa passare attraverso le fasi temporali senza sparire o esplodere.
  \item \emph{Input Node:} Questo nodo si comporta come un normale nodo, prendendo gli input tanto dal resto della rete alla fase temporale precedente quanto dagli input al tempo corrente.
  Solitamente questo nodo viene indicato con $g$ e questa tesi non far\`a eccezione, anche se potrebbe generare confusione dato che sarebbe pi\`u appropiato usare $g$ per indicare i cancelli.
  \`E importante notare che quando usiamo la notazione vettoriale ci riferiamo ad un intero livello di celle.
  Ad esempio, $g$ \`e un vettore che contiene i valori di $g_c$ di tutte le celle in un dato livello.
  Quando, invece, usiamo il pedice $c$, allora ci riferiamo ad una singola cella di memoria.
  \item \emph{Multiplicative Gating:} I ``cancelli'' moltiplicativi (gates) sono una peculiarit\`a dei modelli LSTM.
  Qu\`i un'unit\`a sigmoidale chiamata \emph{gate} viene addestrata dato l'input e una connessione ricorrente in arrivo dal passo temporale precedente.
  Alcuni valori di interesse sono poi moltiplicati per l'output di questa unit\`a.
  Se il cancello da in output 0, allora questo \`e chiuso e il flusso dei dati viene interrotto.
  Se, invece, da in output 1 allora in cancello \`e aperto e il flusso dei dati pu\`o passarci attraverso.
  Il modello LSTM originale prevedeva due cancelli:
  \begin{itemize}
    \item \emph{Input Gate:} Il primo \`e l'\emph{input gate} $i_c$, che \`e moltiplicato per il nodo input $g_c$
    \item \emph{Output Gate:} Il secondo cancello \`e chiamato \emph{output gate} $o_c$.
    Questo viene moltiplicato per il valore dello stato intenro $s_c$ per produrre il valore in output $v_c$ della cella di memoria.
    Questo poi viene dato in pasto al livello nascosto della LSTM nel prossimo passo temporale $h^{(t+1)}$ insieme all'output $\hat{y}^(t)$ generato al passo corrente.
  \end{itemize}
  Questi cancelli vengono indicati con $y_{in}$ e $y_{out}$, tuttavia questa notazione genera confusione perch\`e $y$ \`e solitamente utilizzato per indicare l'output nella letteratura di machine learning.
  Per questo motivo utilizzeremo le notazioni $i$, $o$ ed $f$ per indicare i cancelli di \emph{input}, \emph{output} e \emph{forget} (di cui pareler\`o a breve).
\end{itemize}

\begin{figure}[tp]
  \centering
  \begin{center}
    \includegraphics[width=0.7\textwidth]{./images/memoryCellWithForgetGate.png}
  \end{center}
  \caption{Una cella di memoria del modello LSTM con il \emph{forget gate} cos\`i come descritta da Gers et al. in~\cite{Gers:2000}.}
  \label{fig:memoryCellWithForgetGate}
\end{figure}

Fin da quando \`e stato proposto il modello LSTM originale, molte varianto sono state proposte.
Il \emph{forget gate}, proposti nel 2000 da Gers e Schmidhuber~\cite{Gers:2000}, aggiunge un cancello simile a quelli di input e output che permette, alla rete neurale, di svuotare le informazioni presenti nel \emph{constant error carousel}.
In altre parole, questo ulteriore cancello da alla LSTM la capacit\`a di apprendere quando ``dimenticare'' le informazioni ottenute dalle precendit fasi temporali.
Il \emph{forget gate} \`e diventato di uso comune, un pilastro dei sucessivi lavori sulle LSTM.

Formalmente, la computazione in una LSTM procede in accordo ai seguenti calcoli che devono essere valutati ad ogni fase temporale.
Queste equazioni forniscono l'algoritmo completo per una moderna LSTM, comprensiva di forget gate.
\begin{equation}
  g^{(t)} = \phi(W_{gx}x^{(t)} + W_{ih}h^{(t-1)} + b_g)
\end{equation}
\begin{equation}
  i^{(t)} = \sigma(W_{ix}x^{(t)} + W_{ih}h^{(t-1)} + b_i)
\end{equation}
\begin{equation}
  f^{(t)} = \sigma(W_{fx}x^{(t)} + W_{fh}h^{(t-1)} + b_f)
\end{equation}
\begin{equation}
  o^{(t)} = \sigma(W_{ox}x^{(t)} + W_{oh}h^{(t-1)} + b_o)
\end{equation}
\begin{equation}
  s^{(t)} = g^{(t)} \odot i^{(t)} + s^{(t-1)} \odot f^{(t)}
\end{equation}
\begin{equation}
  h^{(t)} = s^{(t)} \odot o^{(t)}
\end{equation}

dove $\odot$ sta per moltiplicazione elemento per elemento.
Il calcolo di una LSTM semplice, senza forget gate, \`e dato impostando $f^{(t)} = 1$ per ogni $t$.
Seguendi l'ultimo stato dell'arte abbiamo usato la funzione tanh $\phi$ per i nodi di input $g$, nonostante l'implementazione originale del modello LSTM prevedesse l'uso di una funzione sigmoide $\sigma$~\cite{Hochreiter:1997}.
Ancora, $h^{(t-1)}$ \`e un vettore contenente i valori $v_c$ dati in output da ciascuna cella di memoria $c$ del livello nascosto dalla precedente fase temporale.

Intuitivamente, in termini di forward pass dei dati, il modello LSTM pu\`o apprendere quando lasciar passare l'attivazione nello stato interno.
Finttantoch\`e il cancello di input (\emph{input gate}) resta chiuso (da in output 0), nessuna attivazione pu\`o passare.
Allo stesso modo, il cancello di output (\emph{output gate}) apprende quando lasciar uscire il valore dello stato intenro.
Quando entrambi i cancelli sono \emph{chiusi}, l'attivazione rimane confinata, senza subire modifiche ne, tantomeno, influenzare l'output delle fasi temporali intermedie.
In terminidi backward pass, il \emph{constant error carousel} permette al gradiente di poter essere propagato attraverso molte fasi temporali, senza sparire o esplodere.
In quest'ottica, quindi, i cancelli apprendono quando lasciar entrare l'\emph{errore} e quando lasciarlo uscire.
Nella pratica, il modello LSTM ha dimostrato un'abilit\`a superiore nell'apprendere dinepndenze a lungo raggio, rispetto alle normali RNN.
Di conseguenza, questo modello, \`e diventato l'attuale stato dell'arte nel campo delle reti neurali.

\section{Torch7}
\label{sec:torch}
\nocite{Collobert:2011}

Torch7 \`e un framework di cumputazione numerica e una libreria di machine learning che estende \emph{Lua}, progettato per rispondere, principalmente, a tre esigenze:
\begin{enumerate}
  \item deve permettere un \emph{facile sviluppo di algoritmi numerici}
  \item deve essere \emph{facile da estendere} (incluso l'uso di altre librerie)
  \item deve essere \emph{veloce}
\end{enumerate}

\subsection{Perch\`e Lua?}
L'uso di un linguaggio di scripting (interpretato) con \emph{buone API C} sembra essere la scelta migliore per soddisfare il requisito (2).
Innanzitutto, un linguaggio ad alto livello rende lo sviluppo di programmi pi\`u semplice e pi\`u comprensibile rispetto ad uno a basso livello.
In secondo luigo, se il linguaggio di programmazione \`e \emph{interpretato}, diventa pi\`u facile provare velocemente varie idee in maniera interattiva.
In fine, con la presenza di buone API C, il linguaggio di scripting diventa il ``collante'' fra varie librerie in diversi linguaggi: differenti modellazioni dello stesso concetto (provenienti da diverse librerie) possono essere nascoste dietro un unico modello realizzato con il linguaggi di scripting, pur mantenendo tutte le funzionalit\`a di tutte le librerie.

Fra tutti i linguaggi di scripting l'unico che pu\`o soddisfare il vincolo (3) \`e Lua.
Lua \`e, infatti, il linguaggio interpretato pi\`u veloce (con anche il pi\`u veloce compilatore Just In Time (JIT)).
Lua ha anche il vantaggio di essere stato progettato per essere facilmente \emph{incorporato} in applicazioni scritti in C, e fornisce dele ottime API C.

\textbf{Lua} \`e stato pensato per essere usato come un potente e leggero linguaggio di scripting per ogni programma che ne avesse bisogno, ed \`e stato implementato a mo' di una libreria scritta in C.
Citando la pagina web di lua:

\begin{quote}
Lua combines simple procedural syntax with powerful data description constructs based on associative arrays and extensible semantics. Lua is dynamically typed, runs by interpreting bytecode for a register-based virtual machine, and has auto- matic memory management with incremental garbage collection, making it ideal for configuration, scripting, and rapid prototyping.
\end{quote}

Lua offre un discreto supporto alla programmazione object-oriented, alla programmazione funzionale e alla programmazione data-driven.
Il tipo principale di Lua \`e la tabella, che implementa un array associativo in maniera molto efficiente.
Un array associativo altro non \`e che un array i cui indici possono essere non solo numeri, ma anche stringhe o qualsiasi altro dipo di valore che il linguaggio mette a disposizione.
Queste tabelle non hanno una dimensione fissa, possono essere ridimensionate in maniera dinamica, e possono essere usate come ``tabelle virtuali'' per altre tabelle, permettendo, in questo modo, di simulare alcune delle caratteristiche del paradigma object-oriented (come ad esempio l'ereditariet\`a).
Inoltre \`e l'unica struttura dati presente in Lua, ma \`e molto potente e flessibile, viene utilizzata per rappresentare semplici array, insiemi, record, code e molte altre strutture dati, in maniera semplice, unoforme ed efficiente.

\subsection{La Classe Tensor}
Torch7 si basa fortemente sulla sua classe \emph{Tensor} (Tensore), che estende l'insieme dei tipi di base di Lua, fornendo un'efficiente implementazione di un array multi dimensionale.
Gran parte delle librerie scritte per Torch7 usano la sua classe Tensor per rappresentare dati come segnali, immagini, video ecc. .
La libreria Tensor di Torch7 fornisce numerose operazioni (fra cui molte operazioni di algebra lineare), implementate efficientemente in C, che sfruttano l'insieme di istruzioni SSE su piattaforma Intel e, optionalmente, sfruttando le performanti implementazioni BLAS/Lapack (come Intel MKL) per eseguire operazioni di algebra lineare.

Di seguito alcune operazioni standard eseguibili con la classe Tensor:

\lstinputlisting[language={[5.0]Lua}]{snippets/tensor.lua}

Inoltre, cos\`i come in matlab, pi\`u tipi possono coesistere in Torch7 ed \`e semplice passare da un tipo ad un altro:

\lstinputlisting[language={[5.0]Lua}]{snippets/tensor_conversions.lua}

\subsection{Pacchetti utilizzati}

Di seguito la lista delle librerie utilizzate nel coros di questa tes:

\subsubsection{torch}
La libreria principale di Torch7: fornisce la class Tensor, strumenti per la serializzazione e altre funzionalit\`a di base.

\subsubsection{nn}
La libreria \emph{nn} fornisce un insieme di moduli standard per il calcolo di reti neurali, oltre ad un insieme di moduli contenitori che possono essere usati per definire qualsivoglia grafo diretto (aciclico o no).
Tramite una descrizione diretta dell'architettura dei grafi, usando dei moduli che sono componibili, si evita la complessit\`a di utilizzare un parser per interpretare il grafo o di creare un compilatore middle-ware.
In pratica, la maggior parte delle reti sono o sequenziali, o hanno una struttura ad albero semplice con ricorsioni.
Il seguente esempio descrive un percettrone multi livello:

\lstinputlisting[language={[5.0]Lua}]{snippets/multilayer_perceptron.lua}

Dove \lstinline[language={[5.0]Lua}]|nn.Sequential()| \`e un contenitore che definisce una rete neurale sequenziale, mentre \lstinline[language={[5.0]Lua}]|nn.Linera()|, \lstinline[language={[5.0]Lua}]|nn.Tanh()| e \lstinline[language={[5.0]Lua}]|nn.SoftMax()| sono i moduli che definiscono i singoli livelli del percettrone.
Ogni modulo, o contenitore, mette a disposizione delle funzioni standard per calcolare il suo output, e funzioni per calcolare le derivate da propagare indietro verso propri input e i prori parametri interni durante la backward pass.
Data la rete definita in precedenza, un input $X$, il gradiente $\partial E / \partial \hat{Y}$ calcolato da un errore $E$ e da un output generato $\hat{Y}$, queste tre funzioni possono essere eseguiti nel seguente modo:

\lstinputlisting[language={[5.0]Lua}]{snippets/forward_backward_pass.lua}

Dove \lstinline[language={[5.0]Lua}]|loss| \`e una qualche funzione di perdita.

\subsubsection{nngraph}
Un estensione della libreria \emph{nn} che semplifica la creazione di architetture neurali complesse.

\subsubsection{optim}
Una librerie che fornisce implementazioni per la discesa del gradiente, per il metodo del gradiente coniugato e altri algoritmi di ottimizzazione.

\subsection{Efficienza}

Torch7 \`e stato progettato pensando principalmente all'efficienza, infatti sfrutta, quando \`e possibile, l'insieme di istruzione \emph{SSE} e supporta due meccanismi di parallelizzazione: \emph{OpenMP} e \emph{CUDA}.
La libreria Tensor fa un uso massiccio di queste tecniche.
Dal punto di vista dell'utente, abilitare CUDA e OpenMP pu\`o portare ad un massiccio incremento delle prestazioni con qualsiasi script ``Lua'', senza alcuno sforzo aggiuntivo, proprio perch\`e la maggior parte delle librerie dipendono dalla libreria Tensor.
Altre libreria (come la libreria \emph{nn}) fanno comunque uso di OpenMP e CUDA anche quando non fanno uso della libreria Tensor.

\subsubsection{CUDA}
Per gli esperimenti svolti durante questa tesi \`e stata utilizzata la tecnologia CUDA.
CUDA (Compute Unified Device Architecture) \`e un framework proprietario nVidia, utilizzato per programmare i loro processori grafici e sfruttarli per eseguire calcoli general purpose, non legati strettamente alla grafica.
CUDA permette di controllare l'intera gerarchia di memoria disponibile al processore grafico, di cui, i due maggiori componento sono la DRAM, una grande memoria esterna ad alta latenza, e la memoria interna al processore, a bassa latenza ma dell'ordine di qualche KB.
Inoltre permette di controllare anche l'intera gerarchida dei core del processore, nonch\`e i meccanismi di interazione con ciascun core, e con le differenti tipologie di memoria.

Per sfruttare CUDA, Torch7 mette a disposizione un nuovo tipo di tensore: \lstinline[language={[5.0]Lua}]|torch.CudaTensor|.
Una volta creato, questo tensore viene memorizzato nella memora DRAM della GPU.
Tutte le operazioni valide per un normale Tensor lo sono anche per un CudaTensor, che astrae completamente l'uso del processore grafico.


\chapter {Esperimenti}\label{chap:chap_3}
In questo capitolo parler\`o dei dataset utilizzati e di come \`e stato impostato l'esperimento.

\section{Part-of-Speech tagging}
In linguistica, il PoS tagging, \`e un processo che consisite nel raggruppare le parole di una frase in calssi dette, appunto, \emph{Part of Speech} o \emph{classi morfologiche}.
Questo raggruppamento, viene fatto sia in base alla definizione della parola stessa che al contesto in cui si trova.

Nella grammatica tradizionale, esistono un numero limitato di classi morfologiche (sostantivo, verbo, aggettivo, articolo, pronome, avverbo, congiunzione, preposizione, ecc..).

Ad esempio, possiamo classificare la frase \emph{Il cane abbaia.} in questo modo:

\centerline{Il\textbf{/ART} cane\textbf{/SOST} abbaia\textbf{/V} .\textbf{/PUNT}}

Tuttavia, chiaramente, esistono molte pi\`u classi di queste.
Per il pronome, possiamo trattare le forme singolari, plurali e possessive come classi distinte.
In molti linguaggi, inoltre, le parole possono essere distinte in base al loro ``caso'' o in base al genere.

Un esempio di classificazione pi\`u dettagliata pu\`o essere:

\begin{center}
Il\textbf{/ART:m:s}

cane\textbf{/SOST:m:s}

abbaia\textbf{/V:ind:pr:3:s}

.\textbf{/PUNT:sent}
\end{center}

In linguistica sono previste classi morfologiche per vari livelli di dettaglio, in base al modello di classificazione scelto.

Solitamente, in sistemi di PoS tagging computerizzati, vengono adottati modelli di classificazione che prevedono un numero elevato di classi (da 50 o pi\`u) e variano in base alla lingua adottata.
Esistono vari modelli comunemente accettati, quelli utilizzati per gli esperimenti di questa tesi sono:
\begin{itemize}
  \item Penn Treebank Tagset, composto da 45 classi e usato per la lingua inglese.
  \item Tanll Tagset, che conta fino a 328 classi ed \`e usato per la lingua italiana.
\end{itemize}

\section{Dataset utilizzati}
\nocite{Zanchetta:2005}
\nocite{Attardi:2008}

\subsection{CoNLL 2000}

\subsection{Evalita 2009}
\emph{Evalita} nasce grazie all'iniziativa dell'\emph{Italian Association for Computational Linguistics} (ALIC),
ed \`e stato approvato dall'\emph{Italian Association for Artificial Intelligence} (AI*IA)
e dall'\emph{Italian Association for Speech Science} (AISV).
Lo scopo del progetto \`e quello di promuovere lo sviluppo di tecnologie in ambito linguistico,
scritto e parlato, per la lingua italiana, fornendo un ambiente condiviso nel quale
differenti sistemi e approcci, possono essere sviluppati e valutati in maniera consistente.

La diffusione di attivit\`a e di metodologie di valutazione condivise costituisce un passo fondamentale
verso lo sviluppo di risorse e tecnologie di Natural Language Processing. Il buon riscontro
ottenuto da Evalita, sia in termini di partecipanti che in termini di qualit\`a dei risultati ottenuti,
ha dimostranto che vale la pena perseguire tali obiettivi anche per la lingua italiana.
Inoltre, i dati di test e di training per le attivit\`a proposte, vengono resi disponibili alla
comunit\`a scientifica come punto di riferimenti per futuri miglioramenti.

In particolare, per questa tesi, si far\`a riferimenti all'attivit\`a di
PoS-Tagging proposta da Evalita nel 2009.

I dataset forniti dagli organizzatori sono costituiti da articoli tratti dall'edizione
online del giornale \emph{La Repubblica} (http://www.repubblica.it).

L'intero corpus \`e formato da 108,874 parole divise in 3,719 frasi.

Questo \`e stato annotato in pi\`u passaggi: il primo \`e stato portato a termina dal
gruppo di Andrea Baroni, dell'Universit\`a di Bologna, che ha classificato manualmente
l'intero corpus adottando un modello di classificazione con poche classi; successivamente
\`e stato utilizzato \emph{MorphIt!}, uno strumento automatizzato, per assegnare una lista di
possibili classi morfologiche a ciascuna parola; il risultato \`e stato poi convertito, per mezzo
di uno script, nel modello di classificazione \emph{Tanl}.

Infine, l'intero corpus \`e stato revisionato manualmente.

\subsubsection{Formato dei dati}
Il dataset di training \`e formato da un unico file di testo, con codifica UTF-8,
dove ogni riga costituisce un token seguito dalla sua classe, separati da una tabulazione,
secondo il seguente schema:

\begin{center}
  \begin{minipage}{5cm}
    \begin{verbatim}
    <TOKEN_1> <TAG1>
    <TOKEN_2> <TAG2>
    ...
    <TOKEN_N> <TAGN>
    <RIGA VUOTA>
    \end{verbatim}
  \end{minipage}
\end{center}

Al termine di ogni frase \`e presenta una riga vuota. Ad esempio:

\begin{center}
  \begin{minipage}{5cm}
    \begin{verbatim}
      A E
      ben B
      pensarci  Vfc
      , FF
      l'  RDns
      intervista  Sfs
      dell' EAns
      on. SA
      Formica SP
      e' VAip3s
      stata VApsfs
      accolta Vpsfs
      in  E
      genere  Sms
      con E
      disinteresse  Sms
      . FS

    \end{verbatim}
  \end{minipage}
\end{center}

Nell'esempio precedente vengono mostrati alcuni fra i pi\`u comuni problemi di tokenizzazione:
\begin{itemize}
  \item Le abbreviazioni vengono trattate come token (\emph{on.})
  \item Possibili espressioni multi parola non vengono trattate come un unico token (\emph{in\_genere})
  \item I clitici non vengono separati dal token (\emph{pensarci})
\end{itemize}

\subsubsection{Il modello di classificazione Tanl}
Il modello \emph{Tanl} in base alle linee guida di \emph{EAGLES}, uno standard
comunemente accettato dalla comunit\`a NLP, ed \`e derivato dalla classificazione
morfologica adottata dal corpus ISST.

Tanl \`e composto da tre livelli di classi morfologiche, ciascun livello aggiunge maggior
dettaglio alla classificazione.

Il primo livello \`e composto da 14 classi (Tabella~\ref{tab:tanl-coarse}):

\begin{table}[!htbp]
  \centering
  \begin{tabular}{| c || l |}
    \hline
    \thead{Tag} & \thead{Descrizione} \\
    \hline
    A & adjective \\
    B & adverb \\
    C & conjunction \\
    D & determiner \\
    E & preposition \\
    F & punctuation \\
    I & interjection \\
    N & numeral \\
    P & pronoun \\
    R & article \\
    S & noun \\
    T & predeterminer \\
    V & verb \\
    X & residual class \\ \hline
  \end{tabular}
  \caption{Tagset Tanl, primo livello} \label{tab:tanl-coarse}
\end{table}

Il secondo livello del modello Tanl contiene 36 classi, di seguito riportate, con relativi esempi (Tabella~\ref{tab:tanl-fine}):
\begin{longtable}{| c | p{.20\textwidth} | p{.30\textwidth} | p{.40\textwidth} |} \hline
  \thead{Tag} & \thead{Descrizione} & \thead{Esempio} & \thead{Contesto} \\ \hline
  A & adjective & bello, buono, pauroso, ottimo  & \parbox[t]{.40\textwidth}{una \emph{bella} passeggiata\\un \emph{ottimo} attaccante\\una persona \emph{paurosa}}\\ \hline
  AP & possessive adjective & mio, tuo, nostro, loro & \parbox[t]{.40\textwidth}{a \emph{mio} parere\\il \emph{tuo} libro}\\ \hline
  B & adverb & bene, fortemente, malissimo, & \parbox[t]{.40\textwidth}{arrivo \emph{domani}\\sto \emph{bene}}\\ \hline
  BN  & negation adverb & non & \emph{non} sto bene\\ \hline
  CC & coordinative conjunction & e, o, ma & \parbox[t]{.40\textwidth}{i libri \emph{e} i quaderni\\vengo \emph{ma} non rimango}\\ \hline
  CS & subordinative conjunction & mentre, quando & \parbox[t]{.40\textwidth}{\emph{quando} ho finito vengo\\\emph{mentre} scrivevo ho finito l'inchiostro}\\ \hline
  DD  & demonstrative determiner & questo, codesto, quello & \parbox[t]{.40\textwidth}{\emph{questo} denaro\\\emph{quella} famiglia}\\ \hline
  DE & exclamative determiner & che, quale, quanto & \parbox[t]{.40\textwidth}{\emph{che} disastro!\\\emph{quale} catastrofe!}\\ \hline
  DI & indefinite determiner & alcuno, certo, tale, parecchio, qualsiasi & \parbox[t]{.40\textwidth}{\emph{alcune} telefonate\\\emph{parecchi} giornali\\\emph{qualsiasi} persona}\\ \hline
  DQ & interrogative determiner & cui, quale & i \emph{cui} libri\\ \hline
  DR & relative determiner & che, quale, quanto & \parbox[t]{.40\textwidth}{\emph{che} cosa\\\emph{quanta} strada\\\emph{quale} formazione}\\ \hline
  E & preposition & di, a, da, in, su, attraverso, verso, prima\_di & \parbox[t]{.40\textwidth}{\emph{a} casa\\\emph{prima\_di} giorno\\\emph{verso} sera}\\ \hline
  EA & articulated preposition & alla, del, nei & \emph{nel} posto\\ \hline
  FB & balanced punctuation & ( ) [ ] { } - \_ ` & \emph{(}sempre\emph{)}\\ \hline
  FF & comma, hyphen & , - & carta, penna, 30\emph{-}40 persone\\ \hline
  FS & sentence boundary punctuation & . ? ! ... & cosa\emph{?}\\ \hline
  I & interjection & ahim\`e, beh, ecco, grazie & \emph{Beh}, che vuoi?\\ \hline
  N & cardinal number & uno, due, cento, mille, 28, 2000 & \parbox[t]{.40\textwidth}{\emph{due} partite\\\emph{28} anni}\\ \hline
  NO & ordinal number & primo, secondo, centesimo & \emph{secondo} posto\\ \hline
  PC & clitic pronoun &mi, ti, ci, si, te, ne, lo, la, gli & \parbox[t]{.40\textwidth}{me \emph{ne} vado\\\emph{si} sono rotti\\\emph{mi} lavo\\\emph{gli} parlo}\\ \hline
  PD & demonstrative pronoun & questo, quello, costui, ci\`o & \parbox[t]{.40\textwidth}{\emph{quello} di Roma\\\emph{costui} uccide}\\ \hline
  PE & personal pronoun & io, tu, egli, noi, voi & \parbox[t]{.40\textwidth}{\emph{io} parto\\\emph{noi} scriviamo}\\ \hline
  PI & indefinite pronoun & chiunque, ognuno, molto & \parbox[t]{.40\textwidth}{\emph{chiunque} venga\\i diritti di \emph{ognuno}}\\ \hline
  PP & possessive pronoun & mio, tuo, suo, loro, proprio & \parbox[t]{.40\textwidth}{il \emph{mio} \`e qui\\pi\`u bella della \emph{loro}}\\ \hline
  PQ & interrogative pronoun & che, chi, quanto & \parbox[t]{.40\textwidth}{non so \emph{chi} parta\\\emph{quanto} costa?\\\emph{che} ha fatto ieri?}\\ \hline
  PR & relative pronoun & che, cui, quale ci\`o & \parbox[t]{.40\textwidth}{\emph{che} dice\\il \emph{quale} afferma\\a \emph{cui} parlo}\\ \hline
  RD & determinative article & il, lo, la, i, gli, le & \parbox[t]{.40\textwidth}{\emph{il} libro\\\emph{i} gatti}\\ \hline
  RI & indeterminative article & uno, un, una & \parbox[t]{.40\textwidth}{\emph{un} amico\\\emph{una} bambina}\\ \hline
  S & common noun & amico, insegnante, verit\`a & \parbox[t]{.40\textwidth}{l'\emph{amico}\\la \emph{verit\`a}}\\ \hline
  SA & abbreviation & ndr, a.C., d.o.c., km & \parbox[t]{.40\textwidth}{30 \emph{km}\\sesto secolo \emph{a.C.}}\\ \hline
  SP & proper noun & Monica, Pisa, Fiat, Sardegna & \emph{Monica} scrive\\ \hline
  T & predeterminer & tutto, entrambi & \parbox[t]{.40\textwidth}{\emph{tutto} il giorno\\\emph{entrambi} i bambini}\\ \hline
  V & main verb & mangio, passato, camminando & \parbox[t]{.40\textwidth}{\emph{mangio} la sera\\il peggio \`e \emph{passato}\\ho \emph{scritto} una lettera}\\ \hline
  VA & auxiliary verb & avere, essere, venire & \parbox[t]{.40\textwidth}{il peggio \`e \emph{passato}\\ho \emph{scritto} una lettera\\viene \emph{fatto} domani}\\ \hline
  VM & modal verb & volere, potere, dovere, solere & \parbox[t]{.40\textwidth}{non posso \emph{venire}\\vuole \emph{comprare} il libro}\\ \hline
  X & residual class  & it includes formulae, unclassified words, alphabetic symbols and the like & \parbox[t]{.40\textwidth}{distanziare di \emph{43''}\\mi \emph{piacce}}\\ \hline
  \caption{Tagset Tanl, secondo livello} \label{tab:tanl-fine}
\end{longtable}

Nella forma pi\`u completa Tanl conta 328 classi che includono informazioni morfologiche, codificate in questo modo:
\begin{itemize}
  \item \emph{genere}: m (maschile), f (femminile), n (non specificato)
  \item \emph{numero}: s (singolare), p (plurale), n (non specificato)
  \item \emph{persona}: 1 (prima), 2 (seconda), 3 (terza)
  \item \emph{modo}: i (indicativo), m (imperativo), c (congiuntivo), d (condizionale), g (gerundio), f (infinito), p (participio)
  \item \emph{tempo}: p (presente), i (imperfetto), s (passato), f (futuro)
  \item \emph{clitico}: c segnala la presenza di clitici aggiuntivi
\end{itemize}


\chapter {Discussione e sviluppi futuri}\label{chap:chap_4}
In questo capitolo parler\`o dell'algoritmo utilizzato durante la fase di sperimentazione.

Tutti i PoS tagger esistenti, considerano la parola, o pi\`u generalmente il
\emph{token}, come l'unit\`a fondamentale dell'apprendimento. Molti algoritmi
prevedono delle fasi di preprocessing del testo atte a semplificarlo, eliminando
molte delle varianti tipiche di una lingua naturale, come ad esempio la \emph{lemmatizzazione}.

La lemmatizzazione consiste nel ridurre una forma \emph{flessa}\footnote{In
linguistica si chiama \emph{flessione} una qualsiasi variazione morfologica delle
parole realizzata per indicarne i tratti grammaticali o sintattici. Avremo ad
esempio diverse forme flesse di un verbo (\emph{io lavoro}, \emph{tu lavori})
oppure di un nome (\emph{il lavoro}, \emph{i lavori}).} di una parola alla sua
forma canonica (non marcata), detta \emph{lemma}\footnote{In linguistica si dice
lemma la citazione di una parola, ossia quella parola che per convenzione è scelta
per rappresentare tutte le forme di una flessione.}. Esistono numerosi algoritmi
di lemmatizzazione come, ad esempio, \emph{Lovins}~\cite{Lovins:1968} e
\emph{Porter}~\cite{Porter:1980}. Entrambi sono algoritmi di \emph{suffix stripping}
che rimuovono, quindi, i suffissi alle parole a partire da un dizionario di suffissi
comuni.

Altro problema \`e la tokenizzazione, ossia la suddivisione del testo in token.
Processo che all'apparenza sembra di semplice soluzione ma che nasconde delle
difficolt\`a. Prendiamo in considerazione \texttt{aren't}, qual \`e la tokenizzazione
corretta? Possiamo infatti avere \texttt{aren't}, \texttt{arent}, \texttt{are nt}
e \texttt{aren t}.

I dizionari poi, vengono costruiti a partire alle parole presenti nel corpus.
La fase di preprocessing \`e fondamentale nella costruzione di un dizionario
efficiente, altrimenti si avrebbero dizionari enormi con un elevato numero di
parole che, pur avendo quasi lo stesso significato (es. \emph{lavoro}, \emph{lavori})
sono rappresentati in maniera del tutto differente e senza alcuna correlazione.

In questa tesi si cercher\`a di cambiare approccio. L'unit\`a base sar\`a il
singolo carattere, piuttosto che il token. Questo approccio semplifica di molto
la fase di preprocessing del testo, eliminandola del tutto. Non sar\`a pi\`u
necessario ne tokenizzare ne lemmatizzare il corpus in input.

\section{Algoritmo per POS-Tagging basato su caratteri}

La totalit\`a dei corpus taggati, nonch\`e dei tagset esistenti, prevedono dei tag
per le parole e non per i caratteri. Quindi, il primo passo per poter addestrare
con successo un PoS-Tagger basato su caratteri, consiste nel trasformare i dataset
scelti.

La trasformazione \`e stata effettuata in questo modo:
\begin{itemize}
  \item Ogni parola del dataset \`e stata divisa in caratteri.
  \item Ad ogni carattere cos\`i ottenuto, \`e stata assegnata la classe della
        parola di appartenenza.
  \item Alle classi assegnate a ciascun carattere sono stati aggiunti dei suffissi:
  \begin{itemize}
    \item il suffisso \emph{-S} (\emph{S}tart) alla classe del primo carattere
          di ogni parola.
    \item il suffisso \emph{-I} (\emph{I}nner) alla classe degli altri caratteri
          della parola.
  \end{itemize}
\end{itemize}

Ad esempio:

\begin{center}
  \begin{minipage}{5cm}
    \begin{verbatim}
     .
     .
     reckons   VBZ
     .
     .
    \end{verbatim}
  \end{minipage}
\end{center}

diventa

\begin{center}
  \begin{minipage}{5cm}
    \begin{verbatim}
     .
     .
     r   VBZ-S
     e   VBZ-I
     c   VBZ-I
     k   VBZ-I
     o   VBZ-I
     n   VBZ-I
     s   VBZ-I
     .
     .
    \end{verbatim}
  \end{minipage}
\end{center}

Un possibile problema nell'addestrare un PoS-Tagger basato su caratteri, riguarda
la separazione fra le parole. Nel PoS-Tagging tradizionale la separazione fra le
parole \`e implicita, ogni token corrisponde ad una parola. Nel nostro caso, non
essendoci alcun tipo di tokenizzazione, si rischia di perdere un'informazione
importante, ossia dove termina una parola e ne inizia un'altra. Una possibile
soluzione consiste nell'aggiungere, in fase di trasformazione del dataset, un
carattere di spazio fra i caratteri delle singole parole ed assegnare a questo
carattere il tag speciale \emph{S}. Tuttavia, questa soluzione, potrebbe non essere
strettamente necessaria. La rete, infatti, potrebbe apprendere lo stesso questa
informazione, grazie al suffisso dato ai tag (-S indica implicitamente l'inizio
di una parola). Difficile determinare a priori quale sia la scelta migliore,
pertanto si \`e optato per adottarle entrambe addestrando pi\`u reti, met\`a
delle quali addestrate con l'aggiungere il carattere di spazio e l'altra met\`a
senza.

Il passo successivo \`e stato quello di creare due dizionari, uno per i caratteri e
l'altro per i tag. Entrambi i dizionari consistono in un insieme di coppie \texttt{
chiave => valore }, dove il \texttt{vaolre} \`e, in entrambi i dizionari, un numero
intero univoco. La \texttt{chiave}, invece, \`e costituita dal carattere, per
quanto riguarda il dizionario dei caratteri, e dal tag, per quanto riguarda il
dizionario dei tag.

Con i dataset convertiti in una sequenza di caratteri e i dizionari per carattere
e tag, \`e possibile iniziare ad addestrare una rete neurale LSTM descritto nell'
algoritmo seguente (algoritmo~\ref{alg:train}).


\begin{algorithm}[ph] \label{alg:train}
  \SetKwFunction{LSTM}{LSTM}\SetKwFunction{size}{size}
  \SetKwFunction{onehot}{onehot}\SetKwFunction{ClassNLLCriterion}{ClassNLLCriterion}
  \SetKwInOut{Input}{input}\SetKwInOut{Output}{output}

  \Input{il training set $\mathcal{D}$, il dizionario $\mathcal{C}$ dei caratteri,
  il dizionario $\mathcal{T}$ dei tag, epoca massima $\varepsilon$, numero di nodi $j$, numero
  di livelli $l$}
  \Output{l'insieme $\mathcal{W}$ dei pesi della rete addestrata}
  \BlankLine
  \tcc{creo la rete neurale LSTM specificando n. di nodi in input, numero di nodi
  in output, numero di livelli nascosti e numero di nodi per livello nascosto}
  $model \gets $ \LSTM{\size{$\mathcal{C}$}, \size{$\mathcal{T}$},
  $l$, $j$}

  \tcc{la funzione di perdita per problemi di classificazione}
  $criterion \gets $ \ClassNLLCriterion{}

  \For{$i \gets 1$ \textbf{to} $\varepsilon$} {
    \For{$char, tag \in \mathcal{D}$} {
      \tcc{codifica onehot per carattere e tag}
      $x \gets $ \onehot{$\mathcal{C}[char]$}

      $y \gets $ \onehot{$\mathcal{T}[tag]$}

      $prediction \gets model.forward(x)$

      $error \gets criterion.forward(prediction, y)$

      $gradOutputs \gets criterion.backward(error, y)$

      $model.backward(x, gradOutputs)$
    }
  }

  $\mathcal{W} \gets model.weights$

  \tcc{restituisco i pesi della rete neurale.}
\caption{$Addestramento$}
\end{algorithm}

\section{Codifica One-Hot}
Dall'algoritmo si possono notare, in particolare, due aspetti che, a dispetto di
quanto si possa immaginare, sono strettamente collegati:

\begin{itemize}
  \item Il numero di nodi in input e output, per la creazione di una rete LSTM,
        sono, rispettivamente, la dimensione del dizionario $\mathcal{C}$ dei
        caratteri e la dimensione del dizionario $\mathcal{T}$ dei tag
  \item Ogni carattere e ogni tag del dataset \`e stato codificato, prima di essere
        dato in input alla rete neurale, con una codifica \emph{One Hot}.
\end{itemize}

Nella codifica One Hot, ad ogni dato che vogliamo rappresentare, \`e associato
un vettore di $n$ bit. La dimensione di $n$ dipende dal numero di dati differenti
che vogliamo poter rappresentare. Infatti, di questi $n$ bit, solo uno \`e acceso
(1) mentre tutti gli altri sono spenti (0). \`E evidente, quindi, che con $n$ bit
\`e possibile rappresentare al massimo $n$ informazioni diverse.

Prendiamo, ad esempio, la codifica in One Hot dei caratteri presenti nel nostro
dataset. In questo caso, il numero massimo di valori rappresentabili \`e dato
dalla dimensione del dizionario dei tag, quindi $n = \operatorname{length}
(\mathcal{C})$. Supponiamo, per semplicit\`a, di avere solo 4 caratteri nel nostro
dizionario (\ref{eq:dictag}).

\begin{equation} \label{eq:dictag}
  \mathcal{C} = \{ ``a", ``b", ``c", ``d" \}
\end{equation}

Ogni carattere sar\`a quindi rappresentato con un vettore di 4 bit in questo modo:

\begin{equation}
  \vec{a} = [ 1, 0, 0, 0]
\end{equation}
\begin{equation}
  \vec{b} = [ 0, 1, 0, 0]
\end{equation}
\begin{equation}
  \vec{c} = [ 0, 0, 1, 0]
\end{equation}
\begin{equation}
  \vec{d} = [ 0, 0, 0, 1]
\end{equation}

Il numero di nodi di input e output \`e diretta conseguenza della scelta di usare
questa codifica. Infatti, per poter dare in input alla rete vettore $\vec{b}$, ad
esempio, sono necessari 4 nodi di input (Figura~\ref{fig:nnInputExample}).

\begin{figure}[tp]
  \centering
  \begin{center}
    \includegraphics[width=0.7\textwidth]{./images/nnInputExample.png}
  \end{center}
  \caption{il vettore $\vec{b} = [ 0, 1, 0, 0]$ dato in input ad una rete neurale
  con 4 nodi di input}
  \label{fig:nnInputExample}
\end{figure}

In output la rete deve restituire a, sua volta, un tag codificato con codifica
One Hot. Di conseguenza, il numero dei nodi di output \`e dato dal numero di tag
possibili.

Dall'algoritmo proposto si evince, dunque, che il numero di nodi di input \`e pari
al numero di caratteri presenti nel dizionario $\mathcal{C}$ dei caratteri e,
allo stesso modo, il numero dei nodi di output \`e pari al numero di tag nel
dizionario $\mathcal{T}$ dei tag.


\section{Valutazioni}

Dal training set \`e stato estratto un piccolo insieme (\~10\% del training set)
usato come validation set durante tutta la fase di addestramento per monitorarne
l'andamento, calcolando periodicamente, su questo insieme di dati, il \emph{loss}
(il valore dell'errore calcolato dalla funzione di perdita).

Per ogni dataset sono state addestrate 96 reti LSTM, ciascuna delle quali
differisce dalle altre per:

\begin{itemize}
  \item numero di livelli (2, 3, 4, 5)
  \item numero di nodi (128, 256, 512, 1024)
  \item numero di fasi temporali (60, 80, 100)
  \item uso o meno del carattere di spazio
\end{itemize}

Ogni rete neurale \`e stata addestrata fino alla 150\textsuperscript{$\circ$} epoca.

A queste reti \`e stato poi dato in input il test set, un carattere alla volta,
utilizzando lo stesso dizionario calcolato durante la fase di addestramento e con
codifica One Hot.

La rete ha quindi classificato singolarmente tutti i caratteri del test set,
dando a questi utlimi una classe col suffisso \emph{-S} e \emph{-I}. L'output
dato dalla rete per ciascun carattere consiste in un vettore di probabilit\`a.
Ciascun elemento del vettore corrisponde ad una determinata classe, e il valore
dell'elemento corrisponde alla probabilit\`a che quella determinata classe sia la
classe corretta per quel carattere. A partire da questo vettore \`e stata quindi
seleziona la classe corrispondente alla probabilit\`a pi\`u alta. Una volta ottenute
le classi di ciascun carattere, \`e stata estratta la classe della parola di appartenenza
di ogni carattere scegliendola fra la classe di maggioranza dei singoli caratteri.

Ad esempio, i caratteri della parola \texttt{reckons} sono stati classificati
dalla rete neurale in questo modo:

\begin{center}
  \begin{minipage}{5cm}
    \begin{verbatim}
     .
     r   VBZ-S
     e   PRP-I
     c   VBZ-I
     k   VBZ-I
     o   NN-I
     n   VBZ-I
     s   VBD-I
     .
     .
    \end{verbatim}
  \end{minipage}
\end{center}

La classe che si ripete con pi\`u frequenza in questa sequenza di caratteri, a
meno del suffisso, \`e \texttt{VBZ}, di conseguenza l'intera parola \texttt{reckons}
\`e stata classificata come

\begin{center}
  \begin{minipage}{5cm}
    \begin{verbatim}
     reckons   VBZ
    \end{verbatim}
  \end{minipage}
\end{center}



% ----------------------------------------------------------------

\bibliographystyle{alpha} %mettere plain o alpha
\bibliography{bibliography}

% ----------------- Insert index
%\addcontentsline{toc}{chapter}{Index of terms}
%\printindex
%\input{tesi.ind}
\clearpage
\vspace*{6cm}

% ----------------------------------------------------------------
\end{document}
% ----------------------------------------------------------------
